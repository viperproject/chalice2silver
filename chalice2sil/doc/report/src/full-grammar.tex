% !TEX TS-program = xelatex+makeindex+bibtex
% !TEX encoding = UTF-8
% !TEX root = ../chalice2sil_report_klauserc.tex

\section{Full SIL Term and Expression Grammar}\label{apdx:grammar}
\begin{grammar}
<Expr> ::= 'acc' (<Location>, <Term>)
	\alt 'old' ( <Expr> )
	\alt 'unfolding' <Term>.<pred-id> 'by' <Term> 'in' <Expr>
	\alt <Term> == <Term>
	\alt <unary-op> <Expr>
	\alt <binary-op> <Expr>
	\alt <dom-pred-id>(\synrep{\synt{Term}}{,})
	\alt '∀' <logical-var-id> : <DataType> :: (<Expr>)
	\alt '∃' <logical-var-id> : <DataType> :: (<Expr>)
	\alt <GExpr>

<Location> ::= <Term>.<field-id>
	\alt <Term>.<pred-id>
\end{grammar}

\begin{grammar}
<PExpr> ::= 'acc' (<PLocation>, <PTerm>)
	\alt 'unfolding' <PTerm>.<pred-id> 'by' <PTerm> 'in' <PExpr>
	\alt <PTerm> == <PTerm>
	\alt <unary-op> <PExpr>
	\alt <binary-op> <PExpr>
	\alt <dom-pred-id>(\synrep{\synt{PTerm}}{,})
	\alt <GExpr>

<PLocation> ::= <PTerm>.<field-id>
	\alt <PTerm>.<pred-id>
\end{grammar}

\begin{grammar}
<DExpr> ::= <DTerm> == <DTerm>
	\alt <unary-op> <DExpr>
	\alt <binary-op> <DExpr>
	\alt <dom-pred-id>(\synrep{\synt{DTerm}}{,})
	\alt '∀' <logical-var-id> : <DataType> :: (<DExpr>)
	\alt '∃' <logical-var-id> : <DataType> :: (<DExpr>)
	\alt <GExpr>
\end{grammar}

\begin{grammar}
<GExpr> ::= <PExpr> == <PExpr>
	\alt <UnaryOp> <PExpr>
	\alt <BinaryOp> <PExpr>
	\alt <dom-pred-id>(\synrep{\synt{PExpr}}{,})
	\alt 'True'
	\alt 'False'
\end{grammar}

\begin{grammar}
<UnaryOp> ::= '¬'

<BinaryOp> ::= '∧' | '∨' | '$\equiv$' | '⇒'
\end{grammar}
\clearpage
\begin{grammar}
<Term> ::= 'if' <Term> 'then' <Term> 'else' <Term>
	\alt 'old'( <Term> )
	\alt <func-id>( \synrep{\synt{Term}}{,} )
	\alt <dom-func-id>( \synrep{\synt{Term}}{,} )
	\alt 'unfolding' <Term>.<pred-id> 'by' <Term> 'in' <Term>
	\alt (<Term>) : <DataType>
	\alt <Term>.<field-id>
	\alt 'perm'( <Location> )
	\alt 'write'
	\alt '0'
	\alt  'E'
	\alt <GTerm>
\end{grammar}

SIL has three literal constants for permission amounts: \lstinline[language=SIL]!write! denotes the full permission, \lstinline[language=SIL]!0! stands for no permission at all (it is distinct from the integer $0$ literal) and \lstinline[language=SIL]!E! represents a single epsilon of permission (a counting permission, see section \ref{sct:counting-permissions}).

\begin{grammar}
<PTerm> ::= 'if' <PTerm> 'then' <PTerm> 'else' <PTerm>
	\alt <var-id>
	\alt <func-id>( \synrep{\synt{PTerm}}{,} 
	\alt <dom-func-id>( \synrep{\synt{PTerm}}{,} )
	\alt 'unfolding' <PTerm>.<pred-id> 'by' <PTerm> 'in' <PTerm>
	\alt (<PTerm>) : <DataType>
	\alt <PTerm>.<field-id>
	\alt <GTerm>
\end{grammar}

\begin{grammar}
<DTerm> ::= 'if' <DTerm> 'then' <DTerm> 'else' <DTerm>
	\alt <logical-var-id>
	\alt <dom-func-id>( \synrep{\synt{DTerm}}{,} )
	\alt <GTerm>
\end{grammar}

\begin{grammar}
<GTerm> ::= 'if' <GTerm> 'then' <GTerm> 'else' <GTerm>
	\alt \synt{integer-literal}
	\alt  <dom-func-id>( \synrep{\synt{GTerm}}{,} )
\end{grammar}