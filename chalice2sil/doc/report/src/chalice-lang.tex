% !TEX TS-program = xelatex
% !TEX encoding = UTF-8
% !TEX root = ../chalice2sil_report_klauserc.tex

\subsection{Chalice}

\begin{sketch}
\begin{itemize}
\item Goals/general ideas of Chalice: Permission-based, modular
\end{itemize}
\end{sketch}

Chalice is a research programming language with the goal of helping programmers detect bugs in their concurrent programs. As with Spec\#, the programmer provides annotations that specify how they intend the program to behave.
These annotations appear in the form of object invariants, loop invariants and method pre- and postconditions.
A verification tool can take such a Chalice program and check statically that it never violates any of the conditions established by the programmer.

The original implementation of the automatic static program verifier for Chalice generates a program in the intermediate verification language Boogie \cite{ByECD+06}.
A second tool, conveniently also called Boogie, takes this intermediate code and generates verification conditions to be solved by an SMT solver.
In its default configuration, Boogie forwards its output to Z3 \cite{dMB08}.


\begin{sketch}
\begin{itemize}
\item Inhale, Exhale, Havoc, terms used in Boogie implementation
\end{itemize}
\end{sketch}
\begin{sketch}
\begin{itemize}
\item Methods
\end{itemize}
\end{sketch}
\begin{sketch}
\begin{itemize}
\item Fork+Join
\end{itemize}
\end{sketch}
\begin{sketch}
\begin{itemize}
\item Predicates + Functions
\end{itemize}
\end{sketch}
\begin{sketch}
\begin{itemize}
\item Monitors
\end{itemize}
\end{sketch}