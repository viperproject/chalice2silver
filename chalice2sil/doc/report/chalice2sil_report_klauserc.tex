% !TEX TS-program = xelatex
% !TEX encoding = UTF-8

\documentclass[11pt,a4paper]{article} % use larger type; default would be 10pt

\usepackage[fleqn]{amsmath}
\usepackage{amsthm,amsfonts}
\usepackage{amssymb} % Must be included BEFORE unicode!!!
\usepackage{fontspec} % Font selection for XeLaTeX; see fontspec.pdf for documentation
\defaultfontfeatures{Mapping=tex-text} % to support TeX conventions like ``---''
\usepackage{xunicode} % Unicode support for LaTeX character names (accents, European chars, etc)
\usepackage{xltxtra} % Extra customizations for XeLaTeX
\usepackage[math-style=ISO]{unicode-math}
%\usepackage{multicol}
\usepackage[usenames,dvipsnames]{xcolor}
\usepackage{mdframed}


\setmainfont{Cambria} % set the main body font (\textrm), assumes Cambria is installed
\setsansfont{Calibri}
\setmonofont{Consolas}
\setmathfont{Cambria Math}

% other LaTeX packages.....
%\usepackage{savetrees}
\usepackage{fullpage}
%\usepackage{bussproofs}
\usepackage{fancyhdr}
\usepackage{listings}
\usepackage{color}
\setlength{\textwidth}{145mm}


% =====================================================
% ===== META INFORMATION ==============================
% =====================================================

\newcommand{\metaSemester}{FS11}
\newcommand{\metaCourseFull}{Translating Chalice into SIL}
\newcommand{\metaCourseShort}{SIL}
\newcommand{\metaCourse}{\metaCourseFull (\metaCourseShort)}
\newcommand{\metaSectionPrefix}{}
\newcommand{\metaAuthor}{Christian Klauser}
\newcommand{\metaAuthorNethz}{klauserc}
\newcommand{\metaAuthorLegi}{08-919-490}

\newcommand{\metaTitle}{Project Report}

% =====================================================
% =====================================================

\pagestyle{fancy}%
\renewcommand{\headrulewidth}{0pt}%
\renewcommand{\footrulewidth}{0pt}%
%\setlength{\headheight}{14pt}%
%\setlength{\footskip}{10pt}%
\fancyhead[L]{}%
\fancyhead[R]{}%
\fancyfoot[C]{\thepage}%

\usepackage{graphicx} % support the \includegraphics command and options
\usepackage{pdflscape}

\input{lib/haskell.tex}
\lstdefinelanguage{Chalice} {
	basicstyle=\ttfamily,
	sensitive=true,
	morecomment=[l][\color{gray_ulisses}\ttfamily]{//},
	morecomment=[s][\color{gray_ulisses}\ttfamily]{\{/*}{*/\}},
	morestring=[b]",
	stringstyle=\color{red},
	showstringspaces=false,
	numberstyle=\tiny,
	numberblanklines=true,
	showspaces=false,
	breaklines=true,
	showtabs=false,
	emph=
	{[1]
		int,token,seq
	},
	emphstyle={[1]\color{darkgreen}},
	emph=
	{[3]
		class,do,else,if,
		module,while,method,fork,join,channel,requires,ensures,invariant,predicate,function,
		between,lock,share,call,var,returns
	},
	emphstyle={[3]\color{blue}\textbf},
}


%Define colors

\lstdefinelanguage{sil} {
	basicstyle=\ttfamily,
	sensitive=true,
	morecomment=[l][\color{gray_ulisses}]{//},
	morecomment=[s][\color{gray_ulisses}]{/*}{*/},
	morestring=[b]",
	stringstyle=\color{red},
	showstringspaces=false,
	numberstyle=\tiny,
	numberblanklines=true,
	showspaces=false,
	breaklines=true,
	showtabs=false,
	emph=
	{[1]
		Integer,Permission,Map,Pair,Predicate,Sequence,Mu
	},
	emphstyle={[1]\color{darkgreen}},
	emph=
	{[2]
		int,token,seq
	},
	emphstyle={[2]\color{darkgreen}},
	emph=
	{[3]
		program,field,predicate,domain,axiom,function,method,implementation,ref,inhale,exhale,call,assume,assert,perm,acc,full,requires,ensures,null,new,var
	},
	emphstyle={[3]\color{blue}\textbf}
}[strings,emph,comments]

\lstnewenvironment{code}
    {\lstset{}%
      \csname lst@SetFirstLabel\endcsname}
    {\csname lst@SaveFirstLabel\endcsname}
\lstset{
      basicstyle=\small\ttfamily,
      flexiblecolumns=false,
      basewidth={0.5em,0.45em},
      tabsize=2,
      frame=single
%      literate={+}{{$+$}}1 {/}{{$/$}}1 {*}{{$*$}}1 {=}{{$=$}}1
%               {>}{{$>$}}1 {<}{{$<$}}1 {\\}{{$λ$}}1
%               {\\\\}{{\char`\\\char`\\}}1
%               {->}{{$\rightarrow$}}2 {>=}{{$≥$}}2 {<-}{{$\leftarrow$}}2
%               {<=}{{$≤$}}2 {=>}{{$\Rightarrow$}}2 
%               {\ .}{{$\circ$}}2 {\ .\ }{{$\circ$}}2
%               {>>}{{>>}}2 {>>=}{{>>=}}2
%               {|}{{$\mid$}}1               
    }
\lstset{tabsize=2}

%=========================================

\newcommand{\abs}[1]{{\left\lvert#1\right\rvert}}
\newcommand{\norm}[1]{{\left\lVert#1\right\rVert}}
\newcommand{\setw}[2]{\ensuremath{\left\{#1\:\middle|\:#2\right\}}}
\newcommand{\albool}{\ensuremath{\{0,1\}}}
\newcommand{\alab}{\ensuremath{\{a,b\}}}
\newcommand{\NN}{\ensuremath{\mathbb{N}}}
\newcommand{\RR}{\ensuremath{\mathbb{R}}}
\newcommand{\ZZ}{\ensuremath{\mathbb{Z}}}
\newcommand{\QQ}{\ensuremath{\mathbb{Q}}}
\newcommand{\CC}{\ensuremath{\mathbb{C}}}
\newcommand{\KK}{\ensuremath{\mathbb{K}}}
\newcommand{\MM}{\ensuremath{\mathbb{M}}}

\newcommand{\entails}{\ensuremath{\vdash\ }}
\renewcommand{\implies}{\ensuremath{\ \rightarrow\ }}

\newenvironment{sketch}{\begin{mdframed}[backgroundcolor=Salmon,hidealllines=true]}{\end{mdframed}}
\newcommand{\ldbrack}{⟦}
\newcommand{\rdbrack}{⟧}
\newcommand{\ch}[1]{\left\ldbrack{}#1\right\rdbrack_\text{Ch}}
\newcommand{\sil}[1]{\left\ldbrack{}#1\right\rdbrack_\text{SIL}}

%=========================================

\usepackage[parfill]{parskip} 
\usepackage{titlesec}
%\titleformat{\section}[display]{\normalfont\Large\bfseries}{Problem \thesection:}{1em}{}
\titleformat{\section}{\normalfont\Large\bfseries\sffamily}{\metaSectionPrefix \thesection\ }{0.5em}{}
\titleformat{\subsection}{\normalfont\large\bfseries\sffamily}{\thesubsection}{0.5em}{}
\titleformat{\subsubsection}{\normalfont\bfseries\sffamily}{\thesubsubsection}{0.4em}{}
\renewcommand{\thesection}{\arabic{section}.}
\renewcommand{\thesubsection}{\arabic{section}.\arabic{subsection}}
\renewcommand{\thesubsubsection}{\arabic{section}.\arabic{subsection}.\arabic{subsubsection}}

%=========================================

\title{{\Huge Translating Chalice into SIL}\\{\small\ }\\{\huge Problem Description}}
\author{Christian Klauser\\ \texttt{klauserc@student.ethz.ch}}
%\date{} % Activate to display a given date or no date (if empty),
         % otherwise the current date is printed 

\begin{document}
\maketitle

\begin{sketch}
\begin{itemize}
\item Use either \lstinline!acc(x.f,write)! (SIL toString) or \lstinline!acc(x.f,full)! (SIL API), not both!
\item Don't use abbreviated read fraction syntax \lstinline!rd(x.f)!, ever.
\end{itemize}
\end{sketch}

% !TEX TS-program = xelatex+makeindex+bibtex
% !TEX encoding = UTF-8
% !TEX root = ../chalice2sil_report_klauserc.tex

\section{Introduction}
Writing correct computer programs is difficult. 
Writing correct concurrent or parallel computer programs is even more difficult. 
One approach to making sure programs do not contain errors is (automatic) \emph{static verification}, the idea of having a computer prove that a given program fulfils its specification and does not crash.
An example of such a system is \emph{Chalice} \cite{LMS09} (section~\ref{sct:chalice}), a research programming language and matching automatic static verification tool.
Targeting a specialized research language, dedicated to the verification of concurrent programs, however, means that one cannot directly apply the tool to code that is used out in the world.

This is where \emph{Semper}, a project at ETH Zürich, comes into play. 
Its goal is to develop an automatic program verifier for concurrent \emph{Scala} \cite{Scala} programs.
Central to the Semper project is an intermediate program representation for verification called \emph{SIL} (section~\ref{sct:sil}).
Programmers are not intended to use SIL directly, but instead write their programs in an existing programming  language and then use a translator to get an intermediate representation that Semper understands.

The goal of this Bachelor's thesis is to build \emph{Chalice2SIL}, the first such a translator, translating from Chalice to SIL (section~\ref{sct:trans}), in order to gain experience with working with SIL (section~\ref{sct:eval}) and the tools involved in Semper.
As the verification methodology used in Semper is based on the methodology underlying Chalice, Chalice is a good fit for the first ``source language'' to be targeted by Semper.

\section{Background}
	% !TEX TS-program = xelatex
% !TEX encoding = UTF-8
% !TEX root = ../chalice2sil_report_klauserc.tex

\subsection{Chalice}
Chalice is a research programming language with the goal of helping programmers detect bugs in their concurrent programs. 
As with most languages aimed at automatic static verification (e.g, Spec\#), the programmer provides annotations that specify how they intend the program to behave.
These annotations appear in the form of monitor invariants, loop invariants and method pre- and postconditions.
A verification tool can take such a Chalice program and check statically that it never violates any of the conditions established by the programmer.

The original implementation of the automatic static program verifier for Chalice generates a program in the intermediate verification language Boogie \cite{ByECD+06}.
A second tool, conveniently also called Boogie, takes this intermediate code and generates verification conditions to be solved by an SMT solver, such as Z3 \cite{dMB08}.

\begin{lstlisting}[language=Chalice,float,caption={Loop invariants, pre- and post conditions in a Chalice program},label=lst:simpleseqchalice]
class Program {
  method intDiv(a : int, b : int) returns (c : int)
    requires 0 <= a && 0 < b;
    ensures c*b <= a && a < (c+1)*b;
  {
    c := 0;
    var r : int := a;
    while(b <= r)
      invariant 0 <= r && r == (a - c*b)
    {
      r := r - b;
      c := c + 1;
    }
  }
}
\end{lstlisting}

Listing \ref{lst:simpleseqchalice} demonstrates how we can implement integer division and have the verifier ensure that our implementation is correct.
Our solution repeatedly subtracts the denominator \lstinline!b! until the rest \lstinline!r! becomes smaller than \lstinline!b!. 
Because this exact algorithm only works for positive numerators and denominators, the method \lstinline[language=chalice]!requires! that the numerator \lstinline!a! is not negative and that the denominator is strictly positive.

Similarly, we specify what the method is supposed to do: the \lstinline[language=Chalice]!ensures! clause tells the verifier that, when our method is ready to return, the resulting quotient \lstinline!c! must be the largest integer for which $c \cdot b \leq a$ still holds.
If the verifier cannot show that this postcondition holds for all invocations of this method that satisfy the precondition, it will reject the program.

The final bit of annotation in this example is the \lstinline[language=Chalice]!invariant! on the \lstinline[language=Chalice]!while! loop. 
A loop invariant is a predicate that needs to hold immediately before the loop is entered and after every iteration, including the last one where the loop condition is already false.
This annotation helps the verifier understand the effects of the loop without knowing how many iterations of the loop would happen at runtime.

\subsubsection{Permissions}

What sets Chalice apart from other languages for program verification is its handling of concurrent access to heap locations. 
Whenever a thread wants to read from or write to a heap location it requires read or write permissions to that location, respectively.
These permissions only exist for verification and would be erased by compilers for Chalice.

\begin{lstlisting}[language=Chalice,float,caption={Chalice example of object creation and (write) accessibility predicates.},label={lst:chaliceexampleaccnew}]
class Cell { var f : int }
class Program {
 	method clone(c : Cell) returns (d : Cell)
    requires c != null && acc(c.f)
    ensures acc(c.f)
    ensures d != null && acc(d.f) && d.f == c.f
  {
    d := new Cell;
    d.f := c.f;
  }
}
\end{lstlisting}

As an under-approximation of the set of permissions a thread would have at runtime, Chalice tracks permissions for each method invocation (stack frame, activation record). 
That way, the verifier can verify method bodies in complete isolation of one another. 
The programmer thus has to specify which heap locations need to be accessible for each method.

In Listing \ref{lst:chaliceexampleaccnew}, we use \emph{accessibility predicates} of the form \lstinline!acc(receiver.field)! in the method's pre- and postcondition. 
\lstinline!acc(c.f)! in the precondition allows us to refer to \lstinline!c.f! in the method body. 
The accessibility predicates in the postcondition, on the other hand, represent permissions that the method will have to ``return'' to its caller upon completion.
Conceptually, the caller passes the permission requested by the callee's precondition on to the callee.
Similarly, the caller receives the permissions mentioned in the callee's postcondition when the call returns.
As a consequence of verifying each method in isolation, it doesn't matter whether a method is called on the same thread or on a thread of it's own (with the caller waiting for the callee's computation to finish). 
The necessary permissions need to be transferred in both scenarios.

\begin{lstlisting}[float,language=Chalice,caption={Calling \lstinline!Program::clone! (extension of Listing \ref{lst:chaliceexampleaccnew})},label={lst:chaliceuseclonefail},numbers=left]
class Program {  
	//...
  method main()
  {
    var c : Cell :=  new Cell;
    c.f := 5;
    var d : Cell;
    call d := clone(c);
    assert d.f == 5; // will fail, c.f might have changed
  }
}
\end{lstlisting}

Listing \ref{lst:chaliceuseclonefail} demonstrates how our \lstinline!clone! method could be used. 
Unfortunately, the assertion on line 9 will fail, as the verifier has to assume that \lstinline!clone! might have changed the value stored in \lstinline!c.f!.
In Chalice, whenever a method gives away all permissions to a memory location (so that it doesn't even have read-access), it must assume that that location has been changed, the next time it gets to read said location.
While we might change augment the postcondition of \lstinline!clone! with the requirement that \lstinline!c.f == old(c.f)! (the value of \lstinline!c.f! at method return must be the same as it was on method entry), there is a much more elegant solution to this problem: \emph{read-only permissions}.

\subsubsection{Percentage Permissions}
When Chalice was originally created, the programmer could specify read-only permissions as \emph{integer percentages} of the full (write) permission. 
\lstinline!acc(x.f,100)! is the same as \lstinline!acc(x.f)!, i.e. grant read and write access, whereas any other strictly positive percentage \lstinline!acc(x.f,n)! for only grants read access to the heap location \lstinline!x.f! ($n\in\NN, 0 < n < 100$).
The verifier keeps track of the exact amount of permission a method holds to each heap location, so that write-access is restored when a method manages to get 100\% of the permission back together, after having handed out parts of it to other methods or threads.

While percentage permissions are very easy to understand, they have the serious drawback that the number of percentage points of permission a method receives to a certain location, essentially determine the maximum number of threads with (shared) read access that method could spawn.
That is a violation of the procedural abstraction that methods are intended to provide.

\subsubsection{Counting Permissions}
Another drawback of percentage permissions is, that it is difficult to deal with a dynamic number of threads to distribute read access over.
As a solution to that problem, Chalice also introduced \emph{``counting permissions''} that are not limited to just 100 ``pieces'' of permission.
Accessibility predicates using counting permissions are written as \lstinline!acc(x.f,rd(1))! and denote an arbitrarily small but still positive (non-zero) amount of permission $\varepsilon$. 
Permission amounts equal to multiples of $\varepsilon$ can be written as \lstinline!acc(x.f,rd(n))!, but any finite number of epsilon permissions are defined to be still smaller than 1\% of permission.
This also means that a method that holds at least 1\% of permission, can always call a method that only requires $n\cdot{}\varepsilon$ of permission.

Unfortunately, counting permissions (often also referred to as \emph{``epsilon permissions''}) still cause method specifications to leak implementation details. 
An epsilon permission cannot be split up further, thus a method that acquires, say, $2 \varepsilon$ of permission to a heap location cannot spawn more than two threads with read access to that heap location.

\subsubsection{Fractional (Read) Permissions}
In order to regain procedural abstraction \cite{HLMS11} added an entirely new kind of permission to Chalice: the fractional read permission, based on \cite{Boy03}. 
The idea is to allow for ``rational'' fractions of permission because, unlike epsilon or percentage permissions, those could always be divided further. 
Composability can still be an issue, even with rational permissions. 
A method that requires $\tfrac{1}{107}$ of permission could still not be called from a method that only has $\tfrac{1}{137}$, even though the fractions passed around the entire system could almost alway be re-scaled to make that call possible.
Thus, instead of forcing the programmer to choose a fixed amount of permission ahead of time, all accessibility predicates involving fractional permissions are kept \emph{abstract}.

The programmer writes \lstinline!acc(x.f,rd)! to denote an abstract (read-only) accessibility predicate to the heap location \lstinline!x.f!. 
The amount of permission denoted by \lstinline!rd! is not fixed. 
When used in a method specification, the \lstinline!rd! can represent a different amount of permission for each method invocation.

To make fractional permissions actually useful, Chalice applies certain constraints to the amount of permission involved in \lstinline!acc(x.f,rd)!. 
Firstly, fractional read permissions always represent a fraction of the caller's permission. 
When a caller gives away a fractional read permission to a heap location, it will always retain read-access to that location. 
That way, the caller can be sure that the contents of the memory location don't change.
Secondly, a common idiom in Chalice is to have methods that return the exact same permissions they acquired in the precondition back to the caller via the postcondition.
When a method requires \lstinline!acc(x.f,rd)! and then ensures \lstinline!acc(x.f,rd)!, we would want these two amounts of permission to be the same. 
That way, a caller that started out with write access to \lstinline!x.f! gets back the exact amount of permission it gave to our method.

Chalice restricts read fractions in method specifications even further: all fractional read permissions in a method contract, even to different heap locations, refer to the same amount of permission (but that amount can still differ between method invocations).
This restriction accounts for the limited information about aliasing available statically and also makes the implementation of fractional read permissions more straightforward.

\begin{lstlisting}[language=Chalice,float,caption={Corrected example using abstract read permissions},label={lst:chaliceabstractread},numbers=left]
class Cell { var f : int }
class Program {
 	method clone(c : Cell) returns (d : Cell)
    requires c != null && acc(c.f,rd)
    ensures acc(c.f,rd)
    ensures d != null && acc(d.f) && d.f == c.f
  {
    d := new Cell;
    d.f := c.f;
  }

  method main()
  {
    var c : Cell :=  new Cell;
    c.f := 5;
    var d : Cell;
    call d := clone(c);
    assert d.f == 5; // will now succeed
		c.f := 7; // we still have write access
  }
}
\end{lstlisting}

Listing \ref{lst:chaliceabstractread} shows the corrected version of our example above (Listings \ref{lst:chaliceexampleaccnew} and \ref{lst:chaliceuseclonefail}) using (abstract) read permissions (\lstinline!acc(c.f,rd)! in lines 4 and 5). 
Note that we don't need to tell the verifier that \lstinline!c.f! won't change separately, because it uses permissions to determine what locations can be modifed by the method call to \lstinline!clone!.

\subsubsection{Fork-Join}
As a language devoted to encoding concurrent programs, Chalice has a built-in mechanism for creating new threads and waiting for threads to complete in the familiar \emph{fork-join} model.
Replacing the \lstinline[language=Chalice]!call! keyword in a (synchronous) method call with \lstinline[language=Chalice]!fork! causes that method to be executed in a newly spawned thread.
As with a synchronous method call, the caller must satisfy the callee's precondition and will give all permissions mentioned in that precondition.

\begin{lstlisting}[language=Chalice]
fork tok := x.m(argument1, argument2, ..., argumentn);
// do something else
join result1, result2, ..., resultn := tok;
\end{lstlisting}

While just forking off threads might work for some scenarios, most of the time the caller will want to collect the results computed by its worker threads at some point.
To that end, the \lstinline[language=Chalice]!fork! statement will return a \emph{token} that the programmer can use to have the calling method wait for the thread associated with the token to complete.
The permissions mentioned in the postcondition of the method used to spawn off the worker thread will also be transferred back to the caller at that point.

\subsubsection{Information Hiding through functions and predicates}
A major shortcoming of pre- and postconditions as presented so far, is that they often ``leak'' implementation details. 
One example of this happening is the \lstinline!clone! method from listing \ref{lst:chaliceabstractread}. 
It ensures that the values from the old object are copied over to the newly created object, but in the process tells the caller that there is exactly one field, called \lstinline!f! on those objects. 
Should the definition of class \lstinline!Cell! ever change, sifting through the entire program and updating specifications is going to be in order.
What the programmer wanted to say is, that the two objects are ``\emph{equal}''. 

\begin{lstlisting}[language=Chalice,float,caption={Alternative definition of \lstinline!Cell! using functions.},label={lst:chaliceequalsfunc}]
class Cell {
  var f : int
  function equals(o : Cell) : bool
    requires acc(f,rd)
    requires o != null ==> acc(o.f,rd)
  { o != null && f == o.f }
}
\end{lstlisting}

\textbf{Functions} help cut down code repetition and put an abstraction layer between the implementation of a method and its clients. 
Listing \ref{lst:chaliceequalsfunc} presents an alternative definition of \lstinline!Cell! that exposes the equality testing function \lstinline!equals!. 
Below is a corresponding signature for the method \lstinline!clone! that uses this function. 
If we were to add a new field to \lstinline!Cell! now, callers of \lstinline!clone! would no longer see a change in the method's signature.

\begin{lstlisting}[language=Chalice]
method clone(c : Cell) returns (d : Cell)
    requires c != null && acc(c.f,rd)
    ensures acc(c.f,rd)
    ensures d != null && acc(d.f) && c.equals(d)
\end{lstlisting}

Notice how the \lstinline!equals! function does \emph{not} have a postcondition that describes the function's result or ``returns'' permissions back to the caller. 
This is because functions are little more than abbreviations of common expressions. 
In order to be used in pre- and postconditions, they are forbidden from changing any state, which is why the programmer doesn't have explicitly return permissions to the function's caller. 
This happens automatically.

\textbf{Predicates}, on the other hand, are a way to abstract over not just values but also over accessibility. 
Additionally, unlike functions, they are treated as abstract entities unless the programmer explicitly ``unfolds'' them to apply their definition.
When a method requires a predicate in its precondition, it will not automatically get the permissions (and other assertions) ``contained'' in the predicate because at that point, the predicate acts like a black box.
The method can pass the predicate to other methods or threads and it behaves much like a permission to a memory location: it cannot be duplicated and once given away, it's gone.

Given a predicate, the programmer can use the \lstinline[language=Chalice]!unfold! statement to ``trade'' the predicate for its definition. 
The current thread will receive all permissions ``contained'' in the predicate and gets to assume any other assertions associated with the predicate.
After the programmer is done operating on the predicate's contents, they can use \lstinline[language=Chalice]!fold! to ``trade'' access permissions in exchange for the predicate.

\begin{lstlisting}[language=Chalice,float,caption={Using the predicate \lstinline!valid! to hide the representation of \lstinline!Indentation!},label={lst:chalicepredicate}]
class Indentation {
    var count : int;

    predicate valid
    { acc(count) && 0 <= count }

    function getCount() : int
        requires valid;
    { unfolding valid in count }

    method increase(amount : int)
        requires valid && 0 <= amount;
        ensures valid;
        ensures old(getCount()) + amount == getCount();
    {
        unfold valid;
        count := count + amount;
        fold valid;
    }
}
\end{lstlisting}

Listing \ref{lst:chalicepredicate} additionally demonstrates the \lstinline[language=Chalice]!unfolding! expression syntax used to temporarily getting access to the contents of a predicate.

\subsubsection{Monitors (locks)}
Using just fork-join, it is impossible for threads to communicate with one another. 
They can only produce a result and all of their memory writes only become visible when they return the exclusive write permissions back to their caller.
To handle more realistic scenarios, such as concurrent access to a shared queue, Chalice comes with \emph{monitors} that allow for exclusive locking of a shared resource.
For each class, the programmer can define a \emph{monitor invariant} that represents the ``resources'' that the monitor is supposed to manage access to. 
As with predicates, this definition can consist of both accessibility predicates and ordinary boolean assertions.

\begin{lstlisting}[language=Chalice,float,caption={Example of the life-cycle an object can go through in Chalice},label={lst:chalicemoncycle}]
class C {
    var f : int;

    invariant acc(f);

    method main(){
        var c : C := new C;
        c.f := 5;
        share c;
        acquire c; c.f := 7; release c;
				// cannot access c.f here
        acquire c; c.f := 6; unshare c;
        assert c.f == 6;
    }
}
\end{lstlisting}

Initially, objects are not available for locking via the monitor mechanism.
When the programmer \emph{shares} an object with other threads using the \lstinline[language=Chalice]!share! statement, the access permissions associated with the invariant get stored in the monitor (similar to \lstinline[language=Chalice]!fold! for predicates).
Threads that subsequently \lstinline[language=Chalice]!acquire! the lock on this \emph{shared} object will receive the contents of the monitor (similar to \lstinline[language=Chalice]!unfold!).
The object is now \emph{locked} and can be made available to other threads via the \lstinline[language=Chalice]!release! statement (similar to \lstinline[language=Chalice]!fold!, again).
The programmer can also revert the conversion to a \emph{shared} object by using the \lstinline[language=Chalice]!unshare! statement (similar to \lstinline[language=Chalice]!unfold!, again). Listing \ref{lst:chalicemoncycle} demonstrates these statements with a single thread.

As with monitors in Java and C\#, in order to guarantee mutual exclusion, threads that reach an \lstinline[language=Chalice]!acquire! statement are blocked until the monitor can grant them the exclusive lock.
With such a simple blocking mechanism comes the risk of deadlocks (thread 1 waiting for monitor $b$, currently held by thread 2, which is waiting for monitor $a$, currently held by thread 1).

To solve this problem, the Chalice verifier makes sure that locks are acquired in a fixed order.
The programmer can assign a \emph{locking level} to a monitor, ensuring that the lock on that monitor can only be acquired when that locking level is \emph{higher} than the locking level of all other locks held by the current thread.
Whether one locking level is higher than another, is determined by a strict partial order that we denote as $<<$.
The \lstinline[language=Chalice]!share! statement seen above optionally accepts clauses of the form $\text{\lstinline[language=Chalice]!between !}\ldots\text{\lstinline[language=Chalice]!and !}\ldots$, $\text{\lstinline[language=Chalice]!above !}\ldots$ or $\text{\lstinline[language=Chalice]!below !}\ldots$ to constrain the \emph{lock level} at which the monitor is installed.
If such a clause is missing, Chalice chooses \lstinline!above waitlevel!, which means that the lock level is higher than the highest lock level of all locks currently held by the thread (we refer to this maximum as a thread's \emph{wait level}).

\begin{lstlisting}[language=Chalice,float,caption={Example of deadlock-prevention},label={lst:chalicedeadlockprevention}]
class C {
    var f : int;
    invariant acc(f);

    method main() {
        var a := new C;
        share a;
        var b := new C;
        share b above a;

				acquire a; acquire b;
				release b; release a;

        acquire b;
        acquire a; // illegal
    }
}
\end{lstlisting}

In listing \ref{lst:chalicedeadlockprevention}, we create two objects \lstinline!a! and \lstinline!b! and share them. 
The lock level of \lstinline!a! defaults to \lstinline!above waitlevel! and the programmer explicitly declare the lock level of \lstinline!b! to be \lstinline!above a!.
This means that if a thread plans to lock both \lstinline!a! and \lstinline!b!, it will have to first lock on \lstinline!a! then \lstinline!b!.
Should the programmer try to lock objects in the opposite order, on \lstinline[language=Chalice]!acquire a! the thread's wait level would already be at the lock level of \lstinline!b!, which is above \lstinline!a!'s.

Lock levels are implemented via a special field called \lstinline!mu! of type \lstinline!Mu! (the type of lock levels), available on every object. 
The \lstinline!mu! field is assigned during \lstinline[language=Chalice]!share! and \lstinline[language=Chalice]!unshare! operations and needs to be read-able for acquiring the lock.

\subsubsection{Details on the Boogie-based Chalice verifier}
\begin{sketch}
Is ``Inhale, exhale and havoc'' a better title for this section?
\end{sketch}

In order to verify Chalice programs, the Boogie-based verifier permission transfer is modelled by two operations: \lstinline[language=SIL]!inhale! and \lstinline[language=SIL]!exhale!. 
They are essentially the same as \lstinline[language=SIL]!assume! and \lstinline[language=SIL]!assert! but in addition to providing and checking facts, they also model the transfer of permissions.
The argument of an \exhale{} operation is an expression that can contain both traditional (boolean) assertions as well as accessibility predicates. 
Conceptually, \exhale{e} represents the transfer of $e$ to another thread. 
Because verification of Chalice methods is modular, we don't specify or even care about which thread will ``receive'' $e$.
For each \exhale operation, the verifier will check (assert) the boolean predicates and remove permissions mentioned in $e$ from the current thread's set of permissions (usually referred to as the thread's ``\emph{permission mask}'').
The \inhale{e} operation works the opposite way. Access permissions mentioned in $e$ are added to the thread's permission mask and boolean predicates get assumed.

\begin{sketch}
Where is the best place for explaining the idea that as soon as you give away all permissions to a location, anything can happen to that location, from your perspective?
\end{sketch}

	% !TEX TS-program = xelatex+makeindex+bibtex
% !TEX encoding = UTF-8
% !TEX root = ../chalice2sil_report_klauserc.tex

\subsection{Semper Intermediate Language (SIL)}
The Semper Intermediate Language is a verification language aimed at the verification of concurrent programs using a methodology based on Chalice.
As its name suggests, SIL is the intermediate language to be used by the various tools that are part of the semper project.  

Much of SIL's design is oriented around Chalice's core elements: methods, permissions and accessibility predicates. 
This also means that SIL programs are encoded on a much higher level of abstraction than the same programs in less focused verification languages, such as Boogie.
As an example: the Boogie-based verifier for Chalice needs to represent permissions as a pair of integers (the number of epsilons and the percentage) whereas in SIL there is a dedicated and built-in data type and associated value constructor functions for permissions.

In this section, we will give an overview of the syntactical structure of SIL programs, diving into more detail where the design of SIL deviates significantly from Chalice.
At this time SIL is mostly intended as an ``exchange format'' and thus has no fixed semantics associated with it.
Also, SIL doesn't currently have a serialised/text form and SIL programs only exist as syntax trees in memory.
As a result we use our own ad-hoc textual representation for SIL program snippets in this report.

%\begin{grammar}
%<NonTerminal> ::= 'keyword-or-punctuation' <terminal> (<Grouped> | <Alternatives>) 
%	\alt <MoreAlternatives> [ <Optional> ] \{ <ZeroOrMore> \}
%\end{grammar}


\newcommand{\synrep}[2]{\{ #1 #2 $\cdots$ \}}

%\begin{grammar}
%<Repetition> ::= [ <Entity> \{ <Separator> <Entity> \} ]
%
%<Repetition> ::= \synrep{\synt{Entity}}{\synt{Separator}} 
%\end{grammar}

\subsubsection{SIL Program Structure}
Each SIL program has a name (\synt{program-id}) and comes with a number of \emph{domain}, \emph{field}, \emph{function}, \emph{predicate} and \emph{method definitions}.
While SIL is certainly aimed at the verification of object oriented programs, it isn't actually necessary to distinguish between the types of references to objects created from different classes.
As a direct result, fields, functions, predicates and methods are not ``contained'' in any form of class definition.

\begin{grammar}
<Program> ::= 'program' <program-id> \\ 
 \{<Domain>\} \\
\{<Field>\} \\
 \{<Function>\} \\
 \{<Predicate>\} \\
 \{<Method>\} 
\end{grammar}

Field and predicate definitions, apart from the fact that they are no tied to a nominal class, are fairly straightforward. Fields consist of a name and a data type and predicates consist of a name and an expression. 
As with Chalice, this predicate expression can contain both accessibility predicates and ordinary boolean predicates.
Field and predicate names must each be unique within a SIL program.

\begin{grammar}
<Field> :: 'field' <field-id> ':' <DataType>

<Predicate> ::= 'predicate' <pred-id> '=' <Expr>
\end{grammar}

Functions, again, are similar to their Chalice counterparts.
They consist of a name, a parameter list, a result type, some preconditions and an implementation.
Note how an \synt{Expr} is expected for the preconditions and a \synt{Term} for the function's body.
That is SIL distinguishing syntactically between assertions/formulae (\synt{Expr}) and expressions that represent a value (\synt{Term}).
\begin{grammar}
<Function> ::= 'function' <id> ( \synrep{\synt{Param}}{,} ) : <DataType> \\
	<Contract> '=' <Term>

<Param> ::= <id> : <DataType>

<Contract> ::= \{ 'requires' <Expr> \} \{ 'ensures' <Expr> \}
\end{grammar}

Methods in SIL have a name (unique among all methods in the program), input and output parameters and a set of pre- and postconditions. 
Every SIL method always has a parameter called \lstinline[language=SIL]!this! of type \lstinline[language=SIL]!ref! in the first position, which represents the \lstinline[language=Chalice]!this! pointer in object oriented languages. 
Having the \lstinline[language=SIL]!this! pointer as an ordinary parameter makes tools that consume SIL programs a bit simpler.
Each method can have multiple implementations that must all share the exact same parameters, pre- and postconditions.
For source languages with virtual methods, the to-SIL-translator would create a method for each ``method slot'' (vtable slot) and add an implementation for each concrete implementation encountered in the program.
\begin{grammar}
<Method> ::= 'method' <method-id> ( \synrep{\synt{Param}}{,} ) : ( \synrep{\synt{Result}}{,} )\\
 <Contract> \{ <Impl> \}

<Result> ::= <Param>

<Impl> ::= 'implementation' <method-id> <Cfg>
\end{grammar}

Method bodies in SIL are represented as a control-flow graph. 
This is mostly because SIL is intended as a format for exchanging programs between the tools that make up Semper as opposed to an actual computer languages used by humans.
Whether an eventual textual representation would retain this form, is not clear at this point.

Unusual about SIL's control-flow graph is that loops are loops are not flattened into basic blocks but retained as a sort of composite block.
A loop block consists of the loop condition, an invariant and a nested control-flow graph for the loop's body.
\begin{grammar}
<Cfg> ::= '\{' \{ <VarDecl> \} \{ <Block> \} '\}'

<VarDecl> ::= 'var' <var-id> : <DataType>

<Block> ::= <BasicBlock>
	\alt <LoopBlock>

<LoopBlock> ::= 'while' <PExpr> [ 'invariant' Expr ] 'do' <Cfg>

<BasicBlock> ::= <label>: '\{' \{ <Stmt> \} <ControlFlow> '\}'

<ControlFlow> ::= 'goto' <label>
	\alt 'halt'
	\alt 'if' <PExpr> 'then goto' <label> 'else goto' <label>
\end{grammar}

At the end of every block there is a single control-flow statement that indicates how control is transferred to other blocks.

\subsubsection{SIL Statements}

\begin{grammar}
<Stmt> ::= <var-id> ':=' <PTerm>
	\alt <var-id>.<field-id> ':=' <PTerm>
	\alt <var-id> ':= new' <DataType>
	\alt	$\vdots$
\end{grammar}
\begin{grammar}
<Stmt> ::= $\vdots$
	\alt (\synrep{\synt{var-id}}{,}) ':=' <PTerm>.<method-id>(\synrep{\synt{PTerm}}{,})
	\alt $\vdots$
\end{grammar}
\begin{grammar}
<Stmt> ::= $\vdots$
	\alt 'inhale' <Expr>
	\alt 'exhale' <Expr>
	\alt $\vdots$
\end{grammar}
\begin{grammar}
<Stmt> ::= $\vdots$
	\alt 'fold' <Term>.<pred-id> 'by' <Term>
	\alt 'unfold' <Term>.<pred-id>
\end{grammar}

\subsubsection{SIL Expressions and Terms}
\begin{grammar}
<Expr> ::= 'acc' (<Location>, <Term>)
	\alt 'old' ( <Expr> )
	\alt 'unfolding' <Term>.<pred-id> 'by' <Term> 'in' <Expr>
	\alt <Term> == <Term>
	\alt <unary-op> <Expr>
	\alt <binary-op> <Expr>
	\alt <dom-pred-id>(\synrep{\synt{Term}}{,})
	\alt ∀ <logical-var-id> : <DataType> :: (<Expr>)
	\alt ∃ <logical-var-id> : <DataType> :: (<Expr>)

<Location> ::= <Term>.<field-id>
	\alt <Term>.<pred-id>
\end{grammar}

\begin{grammar}
<Term> ::= 'if' <Term> 'then' <Term> 'else' <Term>
	\alt <var-id>
	\alt <logical-var-id>
	\alt 'old'( <Term> )
	\alt <func-id>( \synrep{\synt{Term}}{,} )
	\alt <dom-func-id>( \synrep{\synt{Term}}{,} )
	\alt 'unfolding' <Term>.<pred-id> 'by' <Term> 'in' <Term>
	\alt (<Term>) : <DataType>
	\alt <Term>.<field-id>
	\alt 'perm'( <Location> )
	\alt 'write'
	\alt '0'
	\alt  'E'
	\alt <integer-literal>
\end{grammar}

We simplified the presentation of the term and expression grammar for this section and attached the full rules in appendix \ref{apdx:grammar}.

\subsubsection{SIL Domains and Types}
A data type in SIL is either \lit{ref}, the type of all object references, a domain type or a type variable (only for data types in domain templates).
Object references in SIL are treated as potentially having all fields in the SIL program. In practice, only the fields that a method/function has access to, are relevant.
For statically types programming languages, it's the responsibility of the to-SIL-translator to make sure that input programs are type error free.
\begin{grammar}
<DataType> 
	::=  <var-type>
	\alt <dom-type>
	\alt 'ref'
\end{grammar}

In addition to the built-in value domains for integers, booleans and permissions, SIL allows its users to define their own value domains, with (uninterpreted) constructor functions, predicates over values of that domain and their axioms. Domain definitions can come with type parameters, making them templates for concrete domains (similar to C\# generics).
\begin{grammar}
<Domain> ::= 'domain' <dom-id> [ <DomainParameters> ] '\{' <DomainDef> '\}'

<DomainDef> ::= \{ <DomainFunction >\} \{ <DomainPredicate> \} \{ <DomainAxiom> \}

<DomainFunction> ::= 'function' <dom-func-id> ( \synrep{\synt{DataType}}{,} ) : <DataType>

<DomainPredicate> ::= 'predicate' <dom-pred-id> ( \synrep{\synt{DataType}}{,} )

<DomainAxiom> ::= 'axiom' <id> '=' <DExpr>

<DomainParameters> ::= '[' \synrep{\synt{DataType}}{,} ']'
\end{grammar}


% !TEX TS-program = xelatex
% !TEX encoding = UTF-8
% !TEX root = ../chalice2sil_report_klauserc.tex

\section{Translation of Chalice}\label{sct:trans}

\begin{sketch}
High-level overview + including focus areas
\end{sketch}

\subsection{Fractional Read Permissions}\label{sct:frp}
\begin{sketch}
High-level description of how read-fractions are translated in Boogie \\
SIL call node versus same approach as Boogie-encoding \\
Explain how read-permissions are translated
\end{sketch}

To SIL, permissions are just another data type. 
The SIL prelude only defines a set of constructors (like no permission, full permission) and some operators and predicates (like permission addition, subtraction, equality, comparison). 
In particular, it does not specify how permissions are represented. 
This aligns well with the abstract nature in which fractional permissions are written by the programmer.
Like with previous verification backends for Chalice, concrete permission amounts associated with fractional read permissions (\lstinline!acc(x.f,rd)!) are never chosen but only constrained. 
This also means that two textual occurrences of \lstinline!acc(x.f,rd)! do usually not represent the same amount of permission.

This makes fractional permissions very flexible. 
As long as a thread holds any positive amount of permission to a location, we know that we can give away a smaller fraction to a second thread and thereby enable both threads to read that location.
Unfortunately, that amount of flexibility would also make fractional read permissions very hard to use, since every mention of a read permission could theoretically refer to a different amount of permission.
Chalice, therefore, imposes additional constraints on fractional permissions involved in method contracts, predicates, and monitors.
In the following sections we will describe how Chalice2SIL handles each of these situations.

\subsubsection{Methods and fractional permissions}\label{sct:meth}
In Chalice programs, a very common pattern is that a method ``borrows'' permissions to a set of locations, performs its work and then returns the same amount of permission to the method's caller.
In order to readily support this scenario, the original implementation of fractional permissions in Chalice constrains the various fractions mentioned in a method's pre- and postcondition to a value that is chosen once per call site.

\begin{lstlisting}[float,label=lst:actorref,caption={A call that uses and preserves fractional read permissions.},language=Chalice]
class Actor {
	method main(a : int) returns (r : Register)
		ensures r != null
		ensures acc(r.val)
		ensures t.val == a
	{
		r := new Register;
		r.val := 5;
		call act(r);
		r.val := a; //should still have write access here
	}

	method act(r : Register)
		requires r != null
		requires acc(r.val,rd)
		ensures acc(r.val,rd)
	{ /* ... */ }
}
class Register {
	var val : int;
}
\end{lstlisting}

For verifying the callee in listing \ref{lst:actorref}, the Boogie-based implementation introduces a fresh variable permission variable $k_m$, constrains it to be a read-permission ($0<k_m<\text{full}$) and uses it in pre- and postconditions whenever it encounters the abstract permission amount \lstinline!rd!. 
Of course, $k_m$ remains constant throughout the entire body of a method.

\begin{lstlisting}[float,caption={Handling of fractional read permissions by the Boogie-based Chalice verifier.},label=lst:fraccalleeb]
procedure act(r : Register)
{
	var k_m;
	assume (0 < k_m) && (k_m < Permission$FullFraction);
	// inhale (precondition), using k_m for rd
	...
	// exhale (postcondition), using k_m for rd
}
\end{lstlisting}

Notice how the Boogie-based encoding of Chalice in listing \ref{lst:fraccalleeb} does not make use of the pre- and postcondition mechanism provided by Boogie. 
This is primarily because Boogie does not have a concept of inhaling and exhaling of permissions. 
Not so with SIL, which features pre- and postconditions that are aware of access predicates. 
When you call a method in SIL, the precondition is properly exhaled and the postcondition inhaled afterwards.

However, using SIL preconditions also means that we can't just make up a new variable $k_m$, instead it becomes a ``ghost'' parameter and introduces an additional precondition. This makes a lot of sense, since the value $k_m$ is always specific to one call of a method.

\begin{lstlisting}[float,caption={Handling of fractional read permissions by the Chalice2SIL translator},label=lst:fraccallees,language=SIL]
method Actor::act(r : Register, k_m : Permission)
	requires 0 < k_m && k_m < write
	requires r != null
	requires acc(r.val, k_m)
	ensures acc(r.val, k_m)
{ … }
\end{lstlisting}

\subsubsection{Method calls with fractional permissions}\label{sct:methcall}
Without fractional permissions, synchronously calling a method in SIL is as simple as using the built-in call statement:

\begin{lstlisting}[language=SIL]
call () := Actor::act(r)
\end{lstlisting}

SIL takes care of asserting the precondition, exhaling the associated permissions, havocing the necessary heap locations, inhaling the permissions mentioned by the postcondition and finally assuming said postcondition. Adding support for fractional read permissions now only means providing a call-site specific value $k$, right? 

Unfortunately, this where the high-level nature of SIL becomes an obstruction. 
For each method call-site, we want to introduce a fresh variable $k_c$ that represents the fractional permission amount of permission selected for that particular call. 
Then, we want to constrain it to be smaller than the amount of permissions we hold to each of the locations mentioned with abstract read permissions (\lstinline!rd!). For the simple preconditions above, this is easy to accomplish:

\begin{lstlisting}[language=SIL]
var k_c : Permission;
assume k_c < perm(r.val);
call () := Actor::act(r,k_c);
\end{lstlisting}

The term \lstinline!perm(r.val)! is a native SIL term that represents the amount permission the current thread holds to a particular location. 
Sadly, this simple scheme breaks down when we have to deal with multiple instances of access predicates to the same location.

Chalice dictates that
\begin{lstlisting}[language=SIL]
exhale acc(x.f,rd) && acc(x.f,rd)
\end{lstlisting}
is to be treated as
\begin{lstlisting}[language=SIL]
exhale acc(x.f,rd)
exhale acc(x.f,rd)
\end{lstlisting}

Both exhale statements cause $k_c$ to be constrained to the amount of permission held to $x.f$. 
Since exhale has the “side-effect” of giving away the mentioned permissions, this $k_c$ will be constrained further by the second exhale statement.
Additionally, access predicates can be guarded by implications. 
In that case, the Boogie-based Chalice implementation translates 
\begin{lstlisting}[language=Chalice]
exhale P ==> acc(x.f, rd) 
\end{lstlisting}
as
\begin{lstlisting}[language=Chalice]
if(P) 
{ 
	exhale acc(x.f, rd);
}
\end{lstlisting}

At this point we could have decided not to use SIL's built-in call statement and instead encode synchronous method calls as a series of exhale statements, followed by inhaling the callee's postcondition. 
While that would have been equivalent from a verification perspective, we would still be throwing away information: the original program's call graph.

\begin{sketch}
\begin{lstlisting}[language=Chalice]
method m(r : Register, p : bool)
	requires acc(r.val, rd) && (p ==> acc(r.val, rd))
	...	
\end{lstlisting}
\end{sketch}

In order to still use SIL's call statement, we need to keep track of the “remaining” permissions while constraining $k_c$ without actually giving away these permissions, otherwise the verification of the call statement would fail. 
We cannot simply create a copy of the permission mask as a whole and have exhale operate on that instead. 
SIL at least allows us to look up individual entries of the permission mask via the \lstinline!perm(x.f)! term. 
We use that ability to manually create and maintain a permission map of our own. 

Like the permission mask in the Boogie-encoding of Chalice, this data structure must map heap locations, represented as pairs of an object reference and a field identifier, to permission amounts. 
At this time, SIL has no reified field identifiers. 
So in order to distinguish locations (pair of an object reference and a field), the Chalice2SIL translator assigns a unique integer number to each field in the program. 

The only way to populate this map, is to ``copy'' the current state of the actual permission mask entry by entry via the \lstinline!perm(x.f)! term. 
Unfortunately, we can't do this in one big ``initialization'' block, since some of the object reference expression that occur on the right-hand-side of implications might not be defined outside of that implication. 

We could expand implications in the precondition twice: once for initializing our permission map, and once to actually simulate the exhales and constraining of $k_c$, but there is a more concise way.

We start out with two fresh map variables $m$ and $m_0$. The former, $m$, is the permission map we are going to update while constraining $k_c$, whereas $m_0$ represents the state of the permission map immediately before the method call. 
We let the SIL verifier assume that the two maps are identical initially and later add more information about $m$'s initial state by providing assumptions about $m_0$.

\begin{lstlisting}[language=sil]
var k_c : Permission
var m : Map[Pair[ref, Integer], Permission];
var m_0 : Map[Pair[ref, Integer], Permission];
assume 0 < k_c && 1000*k_c < k_m;
assume m == m_0;
--/* acc(r.val,rd) */
assume m_0[(r,1)] == perm(r.val);
assert 0 < m[(r,1)];
assume k_c < m[(r,1)];
m[(r,1)] := m[(r,1)] - k_c;
// p ==> acc(r.val,rd)
if(p){
	assume m_0[(r,1)] == perm(r.val);
	assert 0 < m[(r,1)];
	assume k_c < m[(r,1)];
	m[(r,1)] := m[(r,1)] - k_c;
}
call () := m(r,p,k_c);
\end{lstlisting}


\subsection{Fork-Join}\label{sct:fj}
\begin{sketch}
Explain translation of fork-join.
Explain how read-fraction tracking works identically to the synchronous case
How context of fork is captured for use in join
Problems with old(.) expressions. 
\end{sketch}

\subsection{Predicates and Functions}\label{sct:pf}
\begin{sketch}
1:1 correspondence between Chalice and SIL
Explain global predicate- and function-Fractions
\end{sketch}

\subsection{Monitors with Deadlock Avoidance}\label{sct:mon}
\begin{sketch}
Explain global monitor fraction
Too high-level for Mu: why establishing the correspondence between x.mu and waitlevel is difficult
Explain the solution: muMap, heldMap and the \$CurrentThread object.
\end{sketch}


% !TEX TS-program = xelatex
% !TEX encoding = UTF-8
% !TEX root = ../chalice2sil_report_klauserc.tex

\section{Evaluation}

\subsection{Chalice2SIL+Silicon versus Sy-whatever}
\begin{sketch}
Compare Chalice2SIL+Silicon performance with Syxc (?) both in terms of execution speed and capabilities
\end{sketch}

\subsection{SIL as a translation target/verification intermediate language}
\begin{sketch}
\begin{itemize}
\item Term/Expression distinction (technical)
\item P-expressions/terms versus non-P-expressions/terms (technical)
\item control flow (need for a while loop construct)
\item Some way of capturing state (more general old expression, "transaction"-like scopes)
\end{itemize}
\end{sketch}


% !TEX TS-program = xelatex
% !TEX encoding = UTF-8
% !TEX root = ../chalice2sil_report_klauserc.tex

\section{Conclusion}


\end{document}
