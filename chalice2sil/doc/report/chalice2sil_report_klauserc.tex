% !TEX TS-program = xelatex+makeindex+bibtex
% !TEX encoding = UTF-8

\documentclass[11pt,a4paper]{article} % use larger type; default would be 10pt

\usepackage[fleqn]{amsmath}
\usepackage{amsthm,amsfonts}
\usepackage{amssymb} % Must be included BEFORE unicode!!!
\usepackage{fontspec} % Font selection for XeLaTeX; see fontspec.pdf for documentation
\defaultfontfeatures{Mapping=tex-text} % to support TeX conventions like ``---''
\usepackage{xunicode} % Unicode support for LaTeX character names (accents, European chars, etc)
\usepackage{xltxtra} % Extra customizations for XeLaTeX
\usepackage[math-style=ISO]{unicode-math}
%\usepackage{multicol}
\usepackage[usenames,dvipsnames]{xcolor}
\usepackage{mdframed}
\usepackage{syntax}


\setmainfont{Cambria} % set the main body font (\textrm), assumes Cambria is installed
\setsansfont{Calibri}
\setmonofont{Consolas}
\setmathfont{Cambria Math}

% other LaTeX packages.....
%\usepackage{savetrees}
\usepackage{fullpage}
%\usepackage{bussproofs}
\usepackage{fancyhdr}
\usepackage{listings}
\usepackage{color}
\usepackage{graphicx} % support the \includegraphics command and options
\usepackage[colorlinks=true,citecolor=darkgreen]{hyperref}
\setlength{\textwidth}{145mm}

% =====================================================
% ===== META INFORMATION ==============================
% =====================================================

\newcommand{\metaSemester}{FS11}
\newcommand{\metaCourseFull}{Translating Chalice into SIL}
\newcommand{\metaCourseShort}{SIL}
\newcommand{\metaCourse}{\metaCourseFull (\metaCourseShort)}
\newcommand{\metaSectionPrefix}{}
\newcommand{\metaAuthor}{Christian Klauser}
\newcommand{\metaAuthorNethz}{klauserc}
\newcommand{\metaAuthorLegi}{08-919-490}

\newcommand{\metaTitle}{Project Report}

% =====================================================
% =====================================================

\pagestyle{fancy}%
\renewcommand{\headrulewidth}{0pt}%
\renewcommand{\footrulewidth}{0pt}%
%\setlength{\headheight}{14pt}%
%\setlength{\footskip}{10pt}%
\fancyhead[L]{}%
\fancyhead[R]{}%
\fancyfoot[C]{\thepage}%


%Define colors
\definecolor{gray_ulisses}{gray}{0.55}
\definecolor{castanho_ulisses}{rgb}{0.71,0.33,0.14}
\definecolor{preto_ulisses}{rgb}{0.41,0.20,0.04}
\definecolor{green_ulises}{rgb}{0.2,0.75,0}
\definecolor{darkgreen}{rgb}{0.11,0.35,0} 

\lstdefinelanguage{HaskellUlisses} {
	basicstyle=\ttfamily,
	sensitive=true,
	morecomment=[l][\color{gray_ulisses}\ttfamily]{--},
	morecomment=[s][\color{gray_ulisses}\ttfamily]{\{-}{-\}},
	morestring=[b]",
	stringstyle=\color{red},
	showstringspaces=false,
	numberstyle=\tiny,
	numberblanklines=true,
	showspaces=false,
	breaklines=true,
	showtabs=false,
	emph=
	{[1]
		FilePath,IOError,abs,acos,acosh,all,and,any,appendFile,approxRational,asTypeOf,asin,
		asinh,atan,atan2,atanh,basicIORun,break,catch,ceiling,chr,compare,concat,concatMap,
		const,cos,cosh,curry,cycle,decodeFloat,denominator,digitToInt,div,divMod,drop,
		dropWhile,either,elem,encodeFloat,enumFrom,enumFromThen,enumFromThenTo,enumFromTo,
		error,even,exp,exponent,fail,filter,flip,floatDigits,floatRadix,floatRange,floor,
		fmap,foldl,foldl1,foldr,foldr1,fromDouble,fromEnum,fromInt,fromInteger,fromIntegral,
		fromRational,fst,gcd,getChar,getContents,getLine,head,id,inRange,index,init,intToDigit,
		interact,ioError,isAlpha,isAlphaNum,isAscii,isControl,isDenormalized,isDigit,isHexDigit,
		isIEEE,isInfinite,isLower,isNaN,isNegativeZero,isOctDigit,isPrint,isSpace,isUpper,iterate,
		last,lcm,length,lex,lexDigits,lexLitChar,lines,log,logBase,lookup,map,mapM,mapM_,max,
		maxBound,maximum,maybe,min,minBound,minimum,mod,negate,not,notElem,null,numerator,odd,
		or,ord,otherwise,pi,pred,primExitWith,print,product,properFraction,putChar,putStr,putStrLn,quot,
		quotRem,range,rangeSize,read,readDec,readFile,readFloat,readHex,readIO,readInt,readList,readLitChar,
		readLn,readOct,readParen,readSigned,reads,readsPrec,realToFrac,recip,rem,repeat,replicate,return,
		reverse,round,scaleFloat,scanl,scanl1,scanr,scanr1,seq,sequence,sequence_,show,showChar,showInt,
		showList,showLitChar,showParen,showSigned,showString,shows,showsPrec,significand,signum,sin,
		sinh,snd,span,splitAt,sqrt,subtract,succ,sum,tail,take,takeWhile,tan,tanh,threadToIOResult,toEnum,
		toInt,toInteger,toLower,toRational,toUpper,truncate,uncurry,undefined,unlines,until,unwords,unzip,
		unzip3,userError,words,writeFile,zip,zip3,zipWith,zipWith3,listArray,doParse
	},
	emphstyle={[1]\color{darkgreen}},
	emph=
	{[2]
		Bool,Char,Double,Either,Float,IO,Integer,Int,Maybe,Ordering,Rational,Ratio,ReadS,ShowS,String,
		Word8,InPacket
	},
	emphstyle={[2]\color{darkgreen}},
	emph=
	{[3]
		case,class,data,deriving,do,else,if,import,in,infixl,infixr,instance,let,
		module,of,primitive,then,type,where
	},
	emphstyle={[3]\color{blue}\textbf},
	emph=
	{[4]
		quot,rem,div,mod,elem,notElem,seq
	},
	emphstyle={[4]\color{darkgreen}},
	emph=
	{[5]
		EQ,False,GT,Just,LT,Left,Nothing,Right,True,Show,Eq,Ord,Num
	},
	emphstyle={[5]\color{darkgreen}}
}

\lstloadlanguages{Haskell}
\lstnewenvironment{code}
    {\lstset{}%
      \csname lst@SetFirstLabel\endcsname}
    {\csname lst@SaveFirstLabel\endcsname}
\lstset{
      basicstyle=\small\ttfamily,
      flexiblecolumns=false,
      basewidth={0.5em,0.45em},
%      literate={+}{{$+$}}1 {/}{{$/$}}1 {*}{{$*$}}1 {=}{{$=$}}1
%               {>}{{$>$}}1 {<}{{$<$}}1 {\\}{{$λ$}}1
%               {\\\\}{{\char`\\\char`\\}}1
%               {->}{{$\rightarrow$}}2 {>=}{{$≥$}}2 {<-}{{$\leftarrow$}}2
%               {<=}{{$≤$}}2 {=>}{{$\Rightarrow$}}2 
%               {\ .}{{$\circ$}}2 {\ .\ }{{$\circ$}}2
%               {>>}{{>>}}2 {>>=}{{>>=}}2
%               {|}{{$\mid$}}1               
    }

\lstdefinelanguage{Chalice} {
	basicstyle=\ttfamily,
	sensitive=true,
	morecomment=[l][\color{gray_ulisses}\ttfamily]{//},
	morecomment=[s][\color{gray_ulisses}\ttfamily]{\{/*}{*/\}},
	morestring=[b]",
	stringstyle=\color{red},
	showstringspaces=false,
	numberstyle=\tiny,
	numberblanklines=true,
	showspaces=false,
	breaklines=true,
	showtabs=false,
	emph=
	{[1]
		int,token,seq
	},
	emphstyle={[1]\color{darkgreen}},
	emph=
	{[3]
		class,do,else,if,
		module,while,method,fork,join,channel,requires,ensures,invariant,predicate,function,
		between,lock,share,call,var,returns,unshare,acquire,release,lockchange,acc,rd,assert,old,assume,null
	},
	emphstyle={[3]\color{blue}\textbf},
}


%Define colors

\lstdefinelanguage{sil} {
	basicstyle=\ttfamily,
	sensitive=true,
	morecomment=[l][\color{gray_ulisses}]{//},
	morecomment=[s][\color{gray_ulisses}]{/*}{*/},
	morestring=[b]",
	stringstyle=\color{red},
	showstringspaces=false,
	numberstyle=\tiny,
	numberblanklines=true,
	showspaces=false,
	breaklines=true,
	showtabs=false,
	emph=
	{[1]
		Integer,Permission,Map,Pair,Predicate,Sequence,Mu
	},
	emphstyle={[1]\color{darkgreen}},
	emph=
	{[2]
		int,token,seq
	},
	emphstyle={[2]\color{darkgreen}},
	emph=
	{[3]
		program,field,predicate,domain,axiom,function,method,implementation,ref,inhale,exhale,call,assume,assert,perm,acc,full,requires,ensures
	},
	emphstyle={[3]\color{blue}\textbf}
}[strings,emph,comments]

\lstnewenvironment{code}
    {\lstset{}%
      \csname lst@SetFirstLabel\endcsname}
    {\csname lst@SaveFirstLabel\endcsname}
\lstset{
      basicstyle=\small\ttfamily,
      flexiblecolumns=false,
      basewidth={0.5em,0.45em},
      tabsize=2,
      frame=single
%      literate={+}{{$+$}}1 {/}{{$/$}}1 {*}{{$*$}}1 {=}{{$=$}}1
%               {>}{{$>$}}1 {<}{{$<$}}1 {\\}{{$λ$}}1
%               {\\\\}{{\char`\\\char`\\}}1
%               {->}{{$\rightarrow$}}2 {>=}{{$≥$}}2 {<-}{{$\leftarrow$}}2
%               {<=}{{$≤$}}2 {=>}{{$\Rightarrow$}}2 
%               {\ .}{{$\circ$}}2 {\ .\ }{{$\circ$}}2
%               {>>}{{>>}}2 {>>=}{{>>=}}2
%               {|}{{$\mid$}}1               
    }
\lstset{tabsize=2}

%=========================================

\newcommand{\abs}[1]{{\left\lvert#1\right\rvert}}
\newcommand{\norm}[1]{{\left\lVert#1\right\rVert}}
\newcommand{\setw}[2]{\ensuremath{\left\{#1\:\middle|\:#2\right\}}}
\newcommand{\albool}{\ensuremath{\{0,1\}}}
\newcommand{\alab}{\ensuremath{\{a,b\}}}
\newcommand{\NN}{\ensuremath{\mathbb{N}}}
\newcommand{\RR}{\ensuremath{\mathbb{R}}}
\newcommand{\ZZ}{\ensuremath{\mathbb{Z}}}
\newcommand{\QQ}{\ensuremath{\mathbb{Q}}}
\newcommand{\CC}{\ensuremath{\mathbb{C}}}
\newcommand{\KK}{\ensuremath{\mathbb{K}}}
\newcommand{\MM}{\ensuremath{\mathbb{M}}}

\newcommand{\entails}{\ensuremath{\vdash\ }}
\renewcommand{\implies}{\ensuremath{\ \rightarrow\ }}

\newenvironment{sketch}{\begin{mdframed}[backgroundcolor=Salmon,hidealllines=true]}{\end{mdframed}}
\newcommand{\ldbrack}{⟦}
\newcommand{\rdbrack}{⟧}
\newcommand{\ch}[1]{\ensuremath{{\left\ldbrack{}#1\right\rdbrack}_\text{Ch}}}
\newcommand{\openSIL}[1]{\left\ldbrack{}#1\right.}
\newcommand{\closeSIL}[1]{{\left.#1\right\rdbrack{}}_\text{SIL}}
\newcommand{\sil}[1]{\ensuremath{{\left\ldbrack{}#1\right\rdbrack}_\text{SIL}}}
\newcommand{\acc}[1]{\ensuremath{\text{\lstinline!acc!}\left(#1\right)}}
\newcommand{\inhale}[1]{\ensuremath{\text{\lstinline[language=SIL]!inhale!}\ #1}}
\newcommand{\exhale}[1]{\ensuremath{\text{\lstinline[language=SIL]!exhale!}\ #1}}
\newcommand{\semicolon}{\ensuremath{\text{\lstinline!; !}}}

%=========================================

\usepackage[parfill]{parskip} 

%=========================================

\title{{\Huge Translating Chalice into SIL}\\{\small\ }\\{\huge Report}}
\author{Christian Klauser\\ \texttt{klauserc@student.ethz.ch}}
%\date{} % Activate to display a given date or no date (if empty),
         % otherwise the current date is printed 

\begin{document}
\maketitle

% !TEX TS-program = xelatex
% !TEX encoding = UTF-8
% !TEX root = ../chalice2sil_report_klauserc.tex

\section{Introduction}
\subsection{Semper Project}

\section{Background}\label{sct:background}
This section briefly presents both \emph{Chalice} (the ``source language'') and \emph{SIL} (the ``target language''), focusing on the aspects that are important for discussing how Chalice2SIL performs its translation.
	% !TEX TS-program = xelatex+makeindex+bibtex
% !TEX encoding = UTF-8
% !TEX root = ../chalice2sil_report_klauserc.tex

\subsection{Chalice}\label{sct:chalice}
Chalice \cite{LMS09,LM09} is a research programming language with the goal of helping programmers detect bugs in their concurrent programs. 
As with most languages aimed at automatic static verification (e.g, Spec\#) \cite{BLS04}, the programmer provides annotations that specify how they intend the program to behave.
These annotations appear in the form of monitor invariants, loop invariants and method pre- and postconditions.
A verification tool can take such a Chalice program and check statically that it never violates any of the conditions established by the programmer.

The original implementation of the automatic static program verifier for Chalice generates a program in the intermediate verification language Boogie \cite{ByECD+06}.
A second tool, conveniently also called Boogie, takes this intermediate code and generates verification conditions to be solved by an SMT solver, such as Z3 \cite{dMB08}.

\begin{lstlisting}[language=Chalice,float,caption={Loop invariants, pre- and post conditions in a Chalice program},label=lst:simpleseqchalice]
class Program {
  method intDiv(a : int, b : int) returns (c : int)
    requires 0 <= a && 0 < b;
    ensures c*b <= a && a < (c+1)*b;
  {
    c := 0;
    var r : int := a;
    while(b <= r)
      invariant 0 <= r && r == (a - c*b)
    {
      r := r - b;
      c := c + 1;
    }
  }
}
\end{lstlisting}

Listing \ref{lst:simpleseqchalice} demonstrates how we can implement integer division and have the verifier ensure that our implementation is correct.
Our solution repeatedly subtracts the denominator \lstinline!b! until the rest \lstinline!r! becomes smaller than \lstinline!b!. 
Because this exact algorithm only works for positive numerators and denominators, the method \lstinline[language=chalice]!requires! that the numerator \lstinline!a! is not negative and that the denominator is strictly positive.

Similarly, we specify what the method is supposed to do: the \lstinline[language=Chalice]!ensures! clause tells the verifier that, when our method is ready to return, the resulting quotient \lstinline!c! must be the largest integer for which $c \cdot b \leq a$ still holds.
If the verifier cannot show that this postcondition holds for all invocations of this method that satisfy the precondition, it will reject the program.

The final bit of annotation in this example is the \lstinline[language=Chalice]!invariant! on the \lstinline[language=Chalice]!while! loop. 
A loop invariant is a predicate that needs to hold immediately before the loop is entered and after every iteration, including the last one, when the loop condition is already false.
This annotation helps the verifier understand the effects of the loop without knowing how many iterations of the loop would happen at runtime.

\subsubsection{Permissions}

What sets Chalice apart from other languages for program verification is its handling of concurrent access to heap locations. 
Whenever a thread wants to read from or write to a heap location it requires \emph{read} or \emph{write} permissions to that location, respectively. 
A thread having \emph{write} permission to a heap location means that that thread holds \emph{all} permissions to that location. 
However, permissions can also be divided up between multiple threads and as long as a thread holds onto a strictly positive amount of permission to a heap location, it can not only read that location, but is also guaranteed that no thread can write to that location, as write-access requires $100\%$ of all permissions to a heap location.
For a thread to have no permissions at all to a heap location means that this location is completely inaccessible. 
Worse yet, it could be changing at any moment, since another thread might hold full permissions to it.
Chalice therefore forbids access to heap locations where the current thread holds no permissions to.

The amount of permission a thread holds over a certain heap location can change over time, and often does.
The main thread of an implementation of a parallel algorithm, for instance, could start with exclusive write-access to the input data structure and then split up its permissions among a number of worker threads.
These worker threads could then all \emph{read} from the input data structure while performing their work, relying on the fact that concurrent accesses to that data structure are safe since no thread can write to it.
After the worker threads have finished, the main thread can collect the permission fractions given out to the workers and combine them back to a full permission, allowing it to write to the input data structure once again, perhaps to update it with the results retrieved from the worker threads.

Almost as a side-effect of this model, the amount of permission a thread has can be used to \emph{``frame''} the set of  existing heap locations that thread can access (a thread can always allocate new heap space).
All these permissions only exist for verification and would be erased by compilers for Chalice.

\begin{lstlisting}[language=Chalice,float,caption={Chalice example of object creation and (write) accessibility predicates.},label={lst:chaliceexampleaccnew}]
class Cell { var f : int }
class Program {
 	method clone(c : Cell) returns (d : Cell)
    requires c != null && acc(c.f)
    ensures acc(c.f)
    ensures d != null && acc(d.f) && d.f == c.f
  {
    d := new Cell;
    d.f := c.f;
  }
}
\end{lstlisting}

As an under-approximation of the set of permissions a thread would have at runtime, Chalice tracks permissions for each method invocation (stack frame, activation record). 
That way, the verifier can verify method bodies in complete isolation from one another. 
The programmer thus has to specify which heap locations need to be accessible for each method.

In Listing \ref{lst:chaliceexampleaccnew}, we use \emph{accessibility predicates} of the form \acc{\text{\texttt{receiver.field}}} in the method's pre- and postcondition. 
\lstinline!acc(c.f)! in the precondition represents a permission, which allows us to refer to \lstinline!c.f! in the method body. 
The accessibility predicates in the postcondition, on the other hand, represent permissions that the method will have to ``return'' to its caller upon completion.
Conceptually, the caller passes the permission requested by the callee's precondition on to the callee.
Similarly, the caller receives the permissions mentioned in the callee's postcondition when the call returns.
As a consequence of verifying each method in isolation, it doesn't matter whether a method is called on the same thread or on a thread of it's own (with the caller waiting for the callee's computation to finish). 
The necessary permissions need to be transferred in both scenarios.

\begin{lstlisting}[float,language=Chalice,caption={Calling \lstinline!Program::clone! (extension of Listing \ref{lst:chaliceexampleaccnew})},label={lst:chaliceuseclonefail},numbers=left]
class Program {  
	//...
  method main()
  {
    var c : Cell :=  new Cell;
    c.f := 5;
    var d : Cell;
    call d := clone(c);
    assert d.f == 5; // will fail, c.f might have changed
  }
}
\end{lstlisting}

Listing \ref{lst:chaliceuseclonefail} demonstrates how our \lstinline!clone! method could be used. 
Unfortunately, the assertion on line 9 will fail, as the verifier has to assume that \lstinline!clone! might have changed the value stored in \lstinline!c.f!.
In Chalice, whenever a method gives away all permissions to a memory location (so that it doesn't even have read-access), it must assume that that location has been changed, the next time it gets to read said location.
While we might augment the postcondition of \lstinline!clone! with the requirement that \lstinline!c.f == old(c.f)! (the value of \lstinline!c.f! at method return must be the same as it was on method entry), there is a much more elegant solution to this problem: \emph{read-only permissions}.

\subsubsection{Percentage Permissions}
When Chalice was originally created, the programmer could specify read-only permissions as \emph{integer percentages} of the full (write) permission. 
\lstinline!acc(x.f,100)! is the same as \lstinline!acc(x.f)!, i.e. grants read and write access, whereas any other strictly positive percentage \lstinline!acc(x.f,n)! (for $n\in\NN, 0 < n < 100$) only grants read access to the heap location \lstinline!x.f!.
The verifier keeps track of the exact amount of permission a method holds to each heap location, so that write-access is restored when a method manages to get 100\% of the permission back together, after having handed out parts of it to other methods or threads.

While percentage permissions are very easy to understand, they have the serious drawback that the number of percentage points of permission a method receives to a certain location, essentially determine the maximum number of threads with (shared) read access that method could spawn.
That is a violation of the procedural abstraction that methods are intended to provide.

\subsubsection{Counting Permissions}\label{sct:counting-permissions}
Another drawback of percentage permissions is that it is difficult to deal with a dynamic number of threads to distribute read access over.
As a solution to that problem, Chalice also introduced \emph{``counting permissions''} that are not limited to just 100 ``pieces'' of permission.
Accessibility predicates using counting permissions are written as \lstinline!acc(x.f,rd(1))! and denote an arbitrarily small but still positive (non-zero) amount of permission $\varepsilon$. 
Permission amounts equal to multiples of $\varepsilon$ can be written as \lstinline!acc(x.f,rd(n))!, but any finite number of epsilon permissions are defined to be still smaller than 1\% of permission.
This also means that a method that holds at least 1\% of permission, can always call a method that only requires $n\cdot{}\varepsilon$ of permission.

Unfortunately, counting permissions (often also referred to as \emph{``epsilon permissions''}) still cause method specifications to leak implementation details. 
An epsilon permission cannot be split up further, thus a method that acquires, say, $2 \varepsilon$ of permission to a heap location cannot spawn more than two threads with read access to that heap location.

\subsubsection{Fractional (Read) Permissions}
In order to regain procedural abstraction \cite{HLMS11} added an entirely new kind of permission to Chalice: the fractional read permission, based on \cite{Boy03}. 
The idea is to allow for ``rational'' fractions of permission because, unlike epsilon or percentage permissions, these can always be divided further. 
Composability can still be an issue, even with concrete rational permissions. 
A method that requires $\tfrac{1}{107}$ of permission could still not be called from a method that only has $\tfrac{1}{137}$, even though the fractions passed around the entire system could almost alway be re-scaled to make that call possible.
Thus, instead of forcing the programmer to choose a fixed amount of permission ahead of time, all accessibility predicates involving fractional permissions are kept \emph{abstract}.

The programmer writes \lstinline!acc(x.f,rd)! to denote an abstract (read-only) accessibility predicate to the heap location \lstinline!x.f!. 
The amount of permission denoted by \lstinline!rd! is not fixed. 
When used in a method specification, the \lstinline!rd! can represent a different amount of permission for each method invocation.

To make abstract fractional permissions actually useful, Chalice applies certain constraints to the amount of permission involved in \lstinline!acc(x.f,rd)!. 
Firstly, fractional read permissions always represent a fraction of the caller's permission. 
When a caller gives away a fractional read permission to a heap location, it will always retain some permission to that location. 
That way, the caller retains read-access and can be sure that the contents of the memory location don't change.
Secondly, a common idiom in Chalice is to have methods that return the exact same permissions they acquired in the precondition back to the caller via the postcondition.
When a method requires \lstinline!acc(x.f,rd)! and then ensures \lstinline!acc(x.f,rd)!, we would want these two amounts of permission to be the same. 
That way, a caller that started out with write access to \lstinline!x.f! gets back the exact amount of permission it gave to our method.

Chalice restricts read fractions in method specifications even further: for each method invocation, all fractional read permissions in the method contract, even to different heap locations, refer to the same amount of permission (but that amount can still differ between method invocations).
This restriction accounts for the limited information about aliasing available statically and also makes the implementation of fractional read permissions more straightforward.


\begin{lstlisting}[language=Chalice,float,caption={Corrected example using abstract read permissions},label={lst:chaliceabstractread},numbers=left]
class Cell { var f : int }
class Program {
 	method clone(c : Cell) returns (d : Cell)
    requires c != null && acc(c.f,rd)
    ensures acc(c.f,rd)
    ensures d != null && acc(d.f) && d.f == c.f
  {
    d := new Cell;
    d.f := c.f;
  }

  method main()
  {
    var c : Cell :=  new Cell;
    c.f := 5;
    var d : Cell;
    call d := clone(c);
    assert d.f == 5; // will now succeed
		c.f := 7; // we still have write access
  }
}
\end{lstlisting}

\begin{lstlisting}[language=Chalice,float,caption={Alternative definition of \lstinline!Cell! using functions.},label={lst:chaliceequalsfunc}]
class Cell {
  var f : int
  function equals(o : Cell) : bool
    requires acc(f,rd)
    requires o != null ==> acc(o.f,rd)
  { o != null && f == o.f }
}
\end{lstlisting}

Listing \ref{lst:chaliceabstractread} shows the corrected version of our example above (Listings \ref{lst:chaliceexampleaccnew} and \ref{lst:chaliceuseclonefail}) using (abstract) read permissions (\lstinline!acc(c.f,rd)! in lines 4 and 5). 
Note that we don't need to tell the verifier that \lstinline!c.f! won't change separately, because it uses the permissions that the caller retained to determine which locations \emph{cannot} be modified by the call.

\subsubsection{Fork-Join}
As a language devoted to encoding concurrent programs, Chalice has a built-in mechanism for creating new threads and waiting for threads to complete in the familiar \emph{fork-join} model.
Replacing the \lstinline[language=Chalice]!call! keyword in a (synchronous) method call with \lstinline[language=Chalice]!fork! causes that method to be executed in a newly spawned thread.
As with a synchronous method call, the caller must satisfy the callee's precondition and will give all permissions mentioned in that precondition.

\clearpage

\begin{lstlisting}[language=Chalice]
fork tok := x.m(argument1, argument2, ..., argumentn);
// do something else
join result1, result2, ..., resultn := tok;
\end{lstlisting}

While just forking off threads might work for some scenarios, most of the time the caller will want to collect the results computed by its worker threads at some point.
To that end, the \lstinline[language=Chalice]!fork! statement returns a \emph{token} that the programmer can use to have the calling method wait for the thread associated with the token to complete.
The permissions mentioned in the postcondition of the method used to spawn off the worker thread will also be transferred back to the caller at that point.

\subsubsection{Information Hiding through functions and predicates}
A major shortcoming of pre- and postconditions as presented so far, is that they often ``leak'' implementation details. 
One example of this happening is the \lstinline!clone! method from listing \ref{lst:chaliceabstractread}. 
It ensures that the values from the old object are copied over to the newly created object, but in the process tells the caller that there is exactly one field, called \lstinline!f! on those objects. 
Should the definition of class \lstinline!Cell! ever change, sifting through the entire program and updating specifications is going to be in order.
What the programmer wanted to say is, that the two objects are ``\emph{equal}''. 

\textbf{Functions} help cut down code repetition and put an abstraction layer between the implementation of a method and its clients. 
Listing \ref{lst:chaliceequalsfunc} presents an alternative definition of \lstinline!Cell! that exposes the equality testing function \lstinline!equals!. 
Below is a corresponding signature for the method \lstinline!clone! that uses this function. 
If we were to add a new field to \lstinline!Cell! now, callers of \lstinline!clone! would no longer see a change in the method's signature.

\begin{lstlisting}[language=Chalice]
method clone(c : Cell) returns (d : Cell)
    requires c != null && acc(c.f,rd)
    ensures acc(c.f,rd)
    ensures d != null && acc(d.f) && c.equals(d)
\end{lstlisting}

Notice how the \lstinline!equals! function does \emph{not} have a postcondition that describes the function's result or ``returns'' permissions back to the caller.  
In order to be used in pre- and postconditions, they are forbidden from changing any state, which is why the programmer doesn't have explicitly return permissions to the function's caller. 
This happens automatically.

\textbf{Predicates}, on the other hand, are a way to abstract over not just values but also over accessibility. 
Additionally, unlike functions, they are treated as abstract entities unless the programmer explicitly ``unfolds'' them to apply their definition.
When a method requires a predicate in its precondition, it will not automatically get the permissions (and other assertions) ``contained'' in the predicate because at that point, the predicate acts like a black box.
The method can pass the predicate to other methods or threads and it behaves much like a permission to a memory location: it cannot be duplicated and once given away, it's gone.

Given a predicate, the programmer can use the \lstinline[language=Chalice]!unfold! statement to ``trade'' the predicate for its definition. 
The current thread will receive all permissions ``contained'' in the predicate and gets to assume any other assertions associated with the predicate.
After the programmer is done operating on the predicate's contents, they can use \lstinline[language=Chalice]!fold! to ``trade'' access permissions in exchange for the predicate.

\begin{lstlisting}[language=Chalice,float,caption={Using the predicate \lstinline!valid! to hide the representation of \lstinline!Indentation!},label={lst:chalicepredicate}]
class Indentation {
    var count : int;

    predicate valid
    { acc(count) && 0 <= count }

    function getCount() : int
        requires valid;
    { unfolding valid in count }

    method increase(amount : int)
        requires valid && 0 <= amount;
        ensures valid;
        ensures old(getCount()) + amount == getCount();
    {
        unfold valid;
        count := count + amount;
        fold valid;
    }
}
\end{lstlisting}

Listing \ref{lst:chalicepredicate} additionally demonstrates the \lstinline[language=Chalice]!unfolding! expression syntax used to temporarily get access to the contents of a predicate during the evaluation of an expression.

\subsubsection{Monitors (locks)}\label{sct:back-monitors}
Using just fork-join, it is impossible for threads to communicate with one another. 
They can only produce a result and all of their memory writes only become visible when they return the exclusive write permissions back to their caller.
To handle more realistic scenarios, such as concurrent access to a shared queue, Chalice comes with \emph{monitors} that allow for exclusive locking of a shared resource.
For each class, the programmer can define a \emph{monitor invariant} that represents the ``resources'' that the monitor is supposed to manage access to. 
As with predicates, this definition can consist of both accessibility predicates and ordinary boolean assertions.

\begin{lstlisting}[language=Chalice,float,caption={Example of the life-cycle an object can go through in Chalice},label={lst:chalicemoncycle}]
class C {
    var f : int;

    invariant acc(f);

    method main(){
        var c : C := new C;
        c.f := 5;
        share c;
        acquire c; c.f := 7; release c;
				// cannot access c.f here
        acquire c; c.f := 6; unshare c;
        assert c.f == 6;
    }
}
\end{lstlisting}

Initially, objects are not available for locking via the monitor mechanism.
When the programmer \emph{shares} an object with other threads using the \lstinline[language=Chalice]!share! statement, the access permissions associated with the invariant get stored in the monitor (similar to \lstinline[language=Chalice]!fold! for predicates).
Threads that subsequently \lstinline[language=Chalice]!acquire! the lock on this \emph{shared} object will receive the contents of the monitor invariant (similarly to an \lstinline[language=Chalice]!unfold! of a predicate).
The object is now \emph{locked} and can be made available to other threads via the \lstinline[language=Chalice]!release! statement (similarly to a \lstinline[language=Chalice]!fold! of a predicate, again).
The programmer can also revert the conversion to a \emph{shared} object by using the \lstinline[language=Chalice]!unshare! statement (similar to \lstinline[language=Chalice]!unfold!, again). Listing \ref{lst:chalicemoncycle} demonstrates these statements with a single thread.

As with monitors in Java and C\#, in order to guarantee mutual exclusion, threads that reach an \lstinline[language=Chalice]!acquire! statement are blocked until the monitor can grant them the exclusive lock.
With such a simple blocking mechanism comes the risk of deadlocks (thread 1 waiting for monitor $b$, currently held by thread 2, which is waiting for monitor $a$, currently held by thread 1).

To solve this problem, the Chalice verifier makes sure that locks are acquired according to a consistent ordering.
The programmer can assign a \emph{locking level} to a monitor, ensuring that the lock on that monitor can only be acquired when that locking level is \emph{higher} than the locking level of all other locks held by the current thread.
Whether one locking level is higher than another, is denoted by a strict partial order that we denote as $<<$.
The \lstinline[language=Chalice]!share! statement seen above optionally accepts clauses of the form $\text{\lstinline[language=Chalice]!between !}\ldots\text{\lstinline[language=Chalice]!and !}\ldots$, $\text{\lstinline[language=Chalice]!above !}\ldots$ or $\text{\lstinline[language=Chalice]!below !}\ldots$ to constrain the \emph{lock level} at which the monitor is installed.
If such a clause is missing, Chalice chooses \lstinline!above waitlevel!, which means that the lock level is higher than the highest lock level of all locks currently held by the thread (we refer to this maximum as a thread's \emph{wait level}).

\begin{lstlisting}[language=Chalice,float,caption={Example of deadlock-prevention},label={lst:chalicedeadlockprevention}]
class C {
    var f : int;
    invariant acc(f);

    method main() {
        var a := new C;
        share a;
        var b := new C;
        share b above a;

				acquire a; acquire b;
				release b; release a;

        acquire b;
        acquire a; // illegal
    }
}
\end{lstlisting}

In listing \ref{lst:chalicedeadlockprevention}, we create two objects \lstinline!a! and \lstinline!b! and share them. 
The lock level of \lstinline!a! defaults to \lstinline!above waitlevel! and the programmer explicitly declares the lock level of \lstinline!b! to be \lstinline!above a!.
This means that if a thread plans to lock both \lstinline!a! and \lstinline!b!, it will have to first lock \lstinline!a! and then \lstinline!b!.
Should the programmer try to lock objects in the opposite order, on \lstinline[language=Chalice]!acquire a! the thread's wait level would already be at the lock level of \lstinline!b!, which is above \lstinline!a!'s; this would result in an error.

Lock levels are implemented via a special field called \lstinline!mu! of type \lstinline!Mu! (the type of lock levels), available on every object. 
The \lstinline!mu! field is assigned during \lstinline[language=Chalice]!share! and \lstinline[language=Chalice]!unshare! operations and needs to be readable in order to acquire the lock.

\subsubsection{Details on the Boogie-based Chalice verifier}
In order to verify Chalice programs, the Boogie-based verifier models permission transfer by two operations: \lstinline[language=SIL]!inhale! and \lstinline[language=SIL]!exhale!. 
They are essentially the same as \lstinline[language=SIL]!assume! and \lstinline[language=SIL]!assert! but in addition to providing and checking facts, they also model the transfer of permissions.
The argument of an \exhale{} operation is an expression that can contain both traditional (boolean) assertions as well as accessibility predicates. 
Conceptually, \exhale{e} represents the transfer of $e$ to another thread. 
Because verification of Chalice methods is modular, we don't specify or even care about which thread will ``receive'' $e$.
For each \exhale operation, the verifier will check (assert) the boolean predicates and remove permissions mentioned in $e$ from the current thread's set of permissions (usually referred to as the thread's ``\emph{permission mask}'').
The \inhale{e} operation works the opposite way. Access permissions mentioned in $e$ are added to the thread's permission mask and boolean predicates get assumed. More advanced features such as method calls and monitors are translated into combinations of \inhale{}, \exhale{}, \lstinline[language=Chalice]!assume! and \lstinline[language=Chalice]!assert! operations.

	% !TEX TS-program = xelatex
% !TEX encoding = UTF-8
% !TEX root = ../chalice2sil_report_klauserc.tex

\section{Semper Intermediate Language (SIL)}

	% !TEX TS-program = xelatex+makeindex+bibtex
% !TEX encoding = UTF-8
% !TEX root = ../chalice2sil_report_klauserc.tex

\subsection{Silicon}\label{sct:silicon}
Silicon is an automated program verifier for SIL programs based on symbolic execution. 
It was derived from Syxc \cite{schwerhoff2011symbolic}, an alternative verifier for Chalice, and adapted to verify SIL instead.
As Silicon is currently the only verifier for SIL, we use it to test Chalice2SIL.


% !TEX TS-program = xelatex
% !TEX encoding = UTF-8
% !TEX root = ../chalice2sil_report_klauserc.tex

\section{Translation of Chalice}\label{sct:trans}

\begin{sketch}
High-level overview + including focus areas
\end{sketch}

\subsection{Fractional Read Permissions}\label{sct:frp}
To SIL, permissions are just another data type. 
The SIL prelude only defines a set of constructors (like no permission, full permission) and some operators and predicates (like permission addition, subtraction, equality, comparison). 
In particular, it does not specify how permissions are represented. 
This aligns well with the abstract nature in which fractional permissions are written by the programmer.
Like with previous verification backends for Chalice, concrete permission amounts associated with fractional read permissions (\lstinline!acc(x.f,rd)!) are never chosen but only constrained. 
This also means that two textual occurrences of \lstinline!acc(x.f,rd)! do usually not represent the same amount of permission.

This makes fractional permissions very flexible. 
As long as a thread holds any positive amount of permission to a location, we know that we can give away a smaller fraction to a second thread and thereby enable both threads to read that location.
Unfortunately, that amount of flexibility would also make fractional read permissions very hard to use, since every mention of a read permission could theoretically refer to a different amount of permission.
Chalice, therefore, imposes additional constraints on fractional permissions involved in method contracts, predicates, and monitors.
In the following sections we will describe how Chalice2SIL handles each of these situations.

\subsubsection{Methods and fractional permissions}\label{sct:meth}
In Chalice programs, a very common pattern is that a method ``borrows'' permissions to a set of locations, performs its work and then returns the same amount of permission to the method's caller.
In order to readily support this scenario, the original implementation of fractional permissions in Chalice constrains the various fractions mentioned in a method's pre- and postcondition to a value that is chosen once per call site.

\begin{lstlisting}[float,label=lst:actorref,caption={A call that uses and preserves fractional read permissions.},language=Chalice]
class Actor {
	method main(a : int) returns (r : Register)
		ensures r != null
		ensures acc(r.val)
		ensures t.val == a
	{
		r := new Register;
		r.val := 5;
		call act(r);
		r.val := a; //should still have write access here
	}

	method act(r : Register)
		requires r != null
		requires acc(r.val,rd)
		ensures acc(r.val,rd)
	{ /* ... */ }
}
class Register {
	var val : int;
}
\end{lstlisting}

For verifying the callee in listing \ref{lst:actorref}, the Boogie-based implementation introduces a fresh variable permission variable $k_m$, constrains it to be a read-permission ($0<k_m<\text{full}$) and uses it in pre- and postconditions whenever it encounters the abstract permission amount \lstinline!rd!. 
Of course, $k_m$ remains constant throughout the entire body of a method.

\begin{lstlisting}[float,caption={Handling of fractional read permissions by the Boogie-based Chalice verifier.},label=lst:fraccalleeb]
procedure act(r : Register)
{
	var k_m;
	assume (0 < k_m) && (k_m < Permission$FullFraction);
	// inhale (precondition), using k_m for rd
	...
	// exhale (postcondition), using k_m for rd
}
\end{lstlisting}

Notice how the Boogie-based encoding of Chalice in listing \ref{lst:fraccalleeb} does not make use of the pre- and postcondition mechanism provided by Boogie. 
This is primarily because Boogie does not have a concept of inhaling and exhaling of permissions. 
Not so with SIL, which features pre- and postconditions that are aware of access predicates. 
When you call a method in SIL, the precondition is properly exhaled and the postcondition inhaled afterwards.

However, using SIL preconditions also means that we can't just make up a new variable $k_m$, instead it becomes a ``ghost'' parameter and introduces an additional precondition. This makes a lot of sense, since the value $k_m$ is always specific to one call of a method.

\begin{lstlisting}[float,caption={Handling of fractional read permissions by the Chalice2SIL translator},label=lst:fraccallees,language=SIL]
method Actor::act(r : Register, k_m : Permission)
	requires 0 < k_m && k_m < write
	requires r != null
	requires acc(r.val, k_m)
	ensures acc(r.val, k_m)
{ … }
\end{lstlisting}

\subsubsection{Method calls with fractional permissions}\label{sct:methcall}
Without fractional permissions, synchronously calling a method in SIL is as simple as using the built-in call statement:

\begin{lstlisting}[language=SIL]
call () := Actor::act(r)
\end{lstlisting}

SIL takes care of asserting the precondition, exhaling the associated permissions, havocing the necessary heap locations, inhaling the permissions mentioned by the postcondition and finally assuming said postcondition. Adding support for fractional read permissions now only means providing a call-site specific value $k$, right? 

Unfortunately, this where the high-level nature of SIL becomes an obstruction. 
For each method call-site, we want to introduce a fresh variable $k_c$ that represents the fractional permission amount of permission selected for that particular call. 
Then, we want to constrain it to be smaller than the amount of permissions we hold to each of the locations mentioned with abstract read permissions (\lstinline!rd!). For the simple preconditions above, this is easy to accomplish:

\begin{lstlisting}[language=SIL]
var k_c : Permission;
assume k_c < perm(r.val);
call () := Actor::act(r,k_c);
\end{lstlisting}

The term \lstinline!perm(r.val)! is a native SIL term that represents the amount permission the current thread holds to a particular location. 
Sadly, this simple scheme breaks down when we have to deal with multiple instances of access predicates to the same location.

Chalice dictates that
\begin{lstlisting}[language=SIL]
exhale acc(x.f,rd) && acc(x.f,rd)
\end{lstlisting}
is to be treated as
\begin{lstlisting}[language=SIL]
exhale acc(x.f,rd)
exhale acc(x.f,rd)
\end{lstlisting}

Both exhale statements cause $k_c$ to be constrained to the amount of permission held to $x.f$. 
Since exhale has the “side-effect” of giving away the mentioned permissions, this $k_c$ will be constrained further by the second exhale statement.
Additionally, access predicates can be guarded by implications. 
In that case, the Boogie-based Chalice implementation translates 
\begin{lstlisting}[language=Chalice]
exhale P ==> acc(x.f, rd) 
\end{lstlisting}
as
\begin{lstlisting}[language=Chalice]
if(P) 
{ 
	exhale acc(x.f, rd);
}
\end{lstlisting}

At this point we could have decided not to use SIL's built-in call statement and instead encode synchronous method calls as a series of exhale statements, followed by inhaling the callee's postcondition. 
While that would have been equivalent from a verification perspective, we would still be throwing away information: the original program's call graph.

\begin{sketch}
\begin{lstlisting}[language=Chalice]
method m(r : Register, p : bool)
	requires acc(r.val, rd) && (p ==> acc(r.val, rd))
	...	
\end{lstlisting}
\end{sketch}

In order to still use SIL's call statement, we need to keep track of the “remaining” permissions while constraining $k_c$ without actually giving away these permissions, otherwise the verification of the call statement would fail. 
We cannot simply create a copy of the permission mask as a whole and have exhale operate on that instead. 
SIL at least allows us to look up individual entries of the permission mask via the \lstinline!perm(x.f)! term. 
We use that ability to manually create and maintain a permission map of our own. 

Like the permission mask in the Boogie-encoding of Chalice, this data structure must map heap locations, represented as pairs of an object reference and a field identifier, to permission amounts. 
At this time, SIL has no reified field identifiers. 
So in order to distinguish locations (pair of an object reference and a field), the Chalice2SIL translator assigns a unique integer number to each field in the program. 

The only way to populate this map, is to ``copy'' the current state of the actual permission mask entry by entry via the \lstinline!perm(x.f)! term. 
Unfortunately, we can't do this in one big ``initialization'' block, since some of the object reference expression that occur on the right-hand-side of implications might not be defined outside of that implication. 

We could expand implications in the precondition twice: once for initializing our permission map, and once to actually simulate the exhales and constraining of $k_c$, but there is a more concise way.

We start out with two fresh map variables $m$ and $m_0$. The former, $m$, is the permission map we are going to update while constraining $k_c$, whereas $m_0$ represents the state of the permission map immediately before the method call. 
We let the SIL verifier assume that the two maps are identical initially and later add more information about $m$'s initial state by providing assumptions about $m_0$.

\begin{lstlisting}[language=sil,float,caption={Translation sketch for a method call involving fractional read permissions and the precondition \lstinline:acc(r.val,rd) && p ==> acc(r.val):},label=lst:rdcall]
var k_c : Permission
var m : Map[Pair[ref, Integer], Permission];
var m_0 : Map[Pair[ref, Integer], Permission];
assume 0 < k_c && 1000*k_c < k_m;
assume m == m_0;
// acc(r.val,rd)
assume m_0[(r,1)] == perm(r.val);
assert 0 < m[(r,1)];
assume k_c < m[(r,1)];
m[(r,1)] := m[(r,1)] - k_c;
// p ==> acc(r.val,rd)
if(p){
	assume m_0[(r,1)] == perm(r.val);
	assert 0 < m[(r,1)];
	assume k_c < m[(r,1)];
	m[(r,1)] := m[(r,1)] - k_c;
}
// finally, the actual call
call () := m(r,p,k_c);
\end{lstlisting}

\begin{sketch}
Formal-ish translation rules for method call here? 
\[
	T\ch{P \land Q} = \sil{T\ch{P}\text{\lstinline!; !}\ T\ch{Q}}
\]
\[
	T\ch{P ⇒ Q} = \sil{\text{\lstinline!if!}\left(T\ch{P}\right) \left\{ T\ch{Q} \right\}}
\]
etc.
\end{sketch}

After $k_c$ is sufficiently constrained, we just emit a call to our target method. 
The SIL verifier will have to 
	exhale the precondition (giving away the permissions it mentions), 
	havoc heap locations that the caller has lost all permissions to, 
	then inhale the postcondition (receiving permissions it mentions) 
	and finally assign results to local variables as necessary.


\subsection{Asynchronous method calls (Fork-Join)}\label{sct:fj}
\begin{sketch}
\begin{itemize}
\item Explain translation of fork-join.
\item Explain how read-fraction tracking works identically to the synchronous case
\item How context of fork is captured for use in join
\item Problems with old(.) expressions. 
\end{itemize}

\end{sketch}
At this time, SIL only provides synchronous call statements. 
We therefore have to fall back to just exhaling the precondition on fork and inhaling the postcondition on join. 
The challenging aspect of verifying asynchronous method calls is establishing the link between a join and the corresponding fork.
Old expressions, in particular, are difficult to capture in SIL without a dedicated call statement.

\subsubsection{Translation of \lstinline!fork!}\label{sct:fjfork}

The translation of the Chalice \lstinline!fork! statement seems, at least at first, relatively straightforward: Exhale the method's precondition and create a token object with a boolean field called ``\lstinline!joinable!'' set to \lstinline!true!.
But how would we then translate the corresponding \lstinline!join! statement(s)? The method's postcondition is formulated in terms of the method's return values and parameters.
In general we no longer have access to the latter.
The \lstinline!join! might happen in a different method, but even if it occurs in the same method as the \lstinline!fork!, the heap and the values of local variables could have changed in the meantime.
Ideally, we could somehow capture the entire program state and store it in or associate it with the token at the \lstinline!fork! statement.
At the \lstinline!join! statement, we would then evaluate (inhale) the method's postcondition in terms of that program state.

Sadly, SIL currently has no such mechanism. 
It does have old expressions for use in postconditions, but they only carry a special meaning in conjunction with the synchronous SIL \lstinline!call! statement.
Fortunately, we don't actually need to capture the entire program heap. 
The set of values missing at the \lstinline!join! site are the arguments and the values of old expressions. 
Since the size of this set is constant and known statically, we can use ghost fields on the token to ``transport'' these values from the \lstinline!fork! site to the \lstinline!join! site.

Chalice2SIL generates one ghost field for each method argument and one ghost field for each \lstinline!old! expression in the method's postcondition. 
Just before the \lstinline!exhale! statement of a \lstinline!join!, it assigns the effective arguments to the argument ghost fields of the token. It then evaluates the old expressions of the method's postcondition and assigns the results to the corresponding ghost fields. 

There is just one more complication to take care of: \lstinline!old! expressions can appear on the right-hand-side of implications, where they might only be defined part of the time (missing permissions and null references). 
Unfortunately, just expanding implications into if-statements, like we did when constraining $k_c$, is not an option because the left-hand-side of the implication could be a return value. 
Instead, we walk over each \lstinline!old! expression and generate a set of conditions that need to be satisfied for the expression to be defined at the \lstinline!fork! site. 

\begin{sketch}
formal-ish set of rules for the ``defined-ness-condition'' (good name for that?)
\[
	D\sil{x.f} = D\sil{x} ∧ \sil{x \neq \text{\lstinline!null!}} ∧ \sil{\text{\lstinline!perm!}\left(x.f\right)}
\]
\end{sketch}

\begin{sketch}
Need to mention that $k_c$ is computed in exactly the same way as for synchronous calls?
\end{sketch}

\begin{lstlisting}[float,caption={Example of fork and join of method with a possibly undefined \lstinline!old! expression.},label=lst:fjexample,language=chalice,numbers=left]
class Cell { var f : int; }
class SuperCell { var cell : Cell; }

class Main {
    method parallel(d : SuperCell) returns (r: bool)
        requires d != null ==> acc(d.cell, rd) && d.cell != null 
				                    && acc(d.cell.f, rd) && d.cell.f == 5
        ensures r == (d != null)
        ensures r ==> old(d.cell.f == 5)
        ensures r ==> (acc(d.cell, rd) && acc(d.cell.f, rd))
    {
          r := d != null;
    }

    method main(d : SuperCell, c : Cell)
        requires acc(d.cell) && acc(c.f)
        ensures acc(d.cell) && acc(c.f)
    {
        var r : bool;
        d.cell := c;
        c.f := 5;
        fork tk := parallel(d)
        assert c.f == 5; // still have read-access
        join r := tk;
        assert r;
    }
}
\end{lstlisting}

\begin{lstlisting}[float,caption={Translation of the \lstinline!fork! statement on line 22 in listing \ref{lst:fjexample}.},label=lst:fjexamplefork,language=sil]
var tk : ref;
tk := new ref;
inhale acc(tk.joinable,write);
tk.joinable := true;
// constrain k_c, the read fraction for this call 
...
// store arguments in token
inhale acc(tk.this,write);
tk.this := this;
inhale acc(tk.d,write);
tk.d := d;
inhale acc(tk.k_m,write);
tk.k_m := k_c;
//store old values in token
inhale acc(tk.old1,write);
if(d != null && 0 < perm(d.cell) && d.cell != null && 0 < perm(d.cell.f)){
	tk.old1 := (d.cell.f == 5);
}
// "perform" the asynchronous call by exhaling the callee's precondition
exhale this != null && 0 < k_c && k_c < write &&
       d != null ==> acc(d.cell, k_c) && d.cell != null 
				          && acc(d.cell.f, k_c) && d.cell.f == 5
\end{lstlisting}
 
\subsubsection{Translation of \lstinline!join!}\label{sct:fjjoin}
With most of the hard work done when the thread was forked, the translation of a \lstinline!join! statement is relatively straightforward.
First, we must assert that the token is still \lstinline!joinable! (we also need write-access to that field in order to set it to false). 
Then we inhale the method's postcondition using the ghost fields on the token as substitutions for the arguments and old expressions. 
Finally, we have to assign the results of the asynchronous computation to the variables indicated by the Chalice programmer.

A detail worth mentioning is the representation of results for the \lstinline!inhale! statement. 
Chalice2SIL also creates ghost fields on the token for results. 
Since a token is only ever joined once, we can safely inhale the permissions to access those result fields. 
Conceptually, by joining with the current thread, the forked thread transfers access to its results along with all other permissions from its postcondition.

Alternatively, we could have used fresh local variables to represent result values. The only advantage that ghost fields provide, is that we \emph{don't} need to introduce new variables.

\begin{lstlisting}[float,caption={Translation of the \lstinline!join! statement on line 24 in listing \ref{lst:fjexample}.},label=lst:fjexamplejoin,language=sil]
exhale tk.joinable // SIL verifier also needs to assert that tk != null
inhale acc(tk.r,write) && tk.r == (tk.d != null)
   &&  tk.r ==> tk.old1
   &&  tk.r ==> acc(tk.d.cell, tk.k_m) && acc(tk.d.cell.f, tk.k_m);
r := tk.r;
tk.joinable := false;
\end{lstlisting}

The accessibility of all the other ghost fields on the token requires a bit more work. 
Naturally, tokens can also be passed to other threads and joined there. 
The requirement that the joining thread has exclusive access to the \lstinline!joinable! field ensures that only one thread can join on a given token. 
Now, while the ghost fields on the token might be invisible to the Chalice programmer, SIL does not distinguish between ghost fields and ordinary fields in any way.
We need to make sure that every method that tries to access any of the ghost fields actually has permissions to do so.

Fortunately, ghost fields on a token are only accessed when we also have permission to access the \lstinline!joinable! field on that token and it is the Chalice programmer's burden to ensure that a thread has this permission when attempting to join on a token. 
If we could somehow link the amount of permission a thread has to each of the ghost fields to the amount of permission it holds to \lstinline!joinable!, we would always end up with a sufficient amount of permission for the ghost fields.

While SIL provides no built-in support for linking fields together accessibility-wise, we can achieve a similar effect by translating every accessibility predicate for \lstinline!joinable! as an accessibility predicate for that \emph{and} all ghost fields (with the same amount of permission for each). That way, we can be sure that whenever a thread holds full permissions to a \lstinline!joinable! field, it also holds full permissions to all ghost fields on the token. More formally, given a token $t$, a permission amount $p$, ghost fields $a_1 \cdots a_k$ (the arguments) and $o_1 \cdots o_n$ (evaluated old expressions), we apply the following transformation:

\begin{align*}
	&\sil{\acc{t.\text{\lstinline!joinable!, p}}} \\
	&\qquad\text{becomes}\qquad \\
	&\left\ldbrack\acc{t.\text{\lstinline!joinable!}, p} \land \acc{t.\text{\lstinline!this!}, p}\ \land \right. \\
  &\land\ \acc{t.a_1, p} \land \acc{t.a_2, p} \land \cdots \land \acc{t.a_k, p}\ \land \\ 
	&\left. \land\ \acc{t.o_1, p} \land \acc{t.o_2, p} \cdots \acc{t.o_n} \right\rdbrack_{\text{SIL}}
\end{align*}

\subsubsection{Limitations of the current fork-join implementation}\label{sct:fjlimits}
Joining a thread seems deceptively simple when done in the same method the thread was originally forked from. 
This is because the verifier has seen the assignments to the token ghost fields first hand. 
When a thread is joined in a separate method, however, that context is not available because both Silicon and the Boogie-based implementation verify each method in complete isolation.

For just joining a thread in a separate method, the programmer needs to pass both the token and write access to the token's \lstinline!joinable! field to the method that performs the joining 
and ensure that the thread has not been joined already.
Unfortunately, the postcondition of an asynchronous method call joined this way is next to useless, because the verifier has no information about the context of the method call. 
Specifically, the verifier doesn't know anything about the receiver or any of the arguments originally passed to the thread. 
As a consequence, any clause of the postcondition that mentions the \lstinline!this! pointer or an argument is useless to the verifier.

\begin{lstlisting}[float,caption={Limitations with joining in separate methods},label=lst:joinseparatethis,language=chalice]
class Main{
    var f : int;
    method work()
        requires acc(this.f)
        ensures acc(this.f)
    {
    }

    method main()
        requires acc(this.f)
        ensures acc(this.f)
    {
        fork tk := work();
        call client(tk, this);
    }

    method client(tk : token<Main.work>, obj : Main)
        requires acc(tk.joinable) && tk.joinable
        ensures acc(obj.f) // might not hold
    {
        join tk;
    }
}
\end{lstlisting}

Listing \ref{lst:joinseparatethis} demonstrates a simple program that fails to verify because the context of the forked thread is lost when the token is transferred to the callee (\lstinline!client!). 
The verifier will complain that there might not be enough permission to satisfy \lstinline!acc(obj.f)!, because it doesn't know that the \lstinline!this! pointer used to call \lstinline!work! refers to the same object as \lstinline!obj!.
We would like to tell the verifier more about how our token was created. 

\begin{lstlisting}[language=chalice]
        requires tk.thisPtr == obj //not valid Chalice expression
\end{lstlisting}

While the previous example is not valid Chalice code, there is a mechanism that can be used to create similar specifications. Listing \ref{lst:joineval} shows how the \lstinline!eval! expression can be used to provide the verifier with the information necessary to prove that the method satisfies its postcondition. 

\begin{lstlisting}[float,caption={\lstinline!eval! expression in Chalice},label=lst:joineval,language=chalice]
method client(tk : token<Main.work>, obj : Main)
        requires acc(tk.joinable) && tk.joinable
        requires eval(tk.fork this.work(), this == obj)
        ensures acc(obj.f)
    { join tk; }
\end{lstlisting}

An $\ch{\text{\lstinline!eval!}\left(r.a, e\right)}$ expression consists of three parts: the ``context'' $c$ (the token in our case), the description of the ``\emph{eval state}'' $a$ and an expression $e$ to be evaluated in that state. In our case, we specify a ``\emph{call state}'' of the form $\ch{\text{\lstinline!fork!}\ r.m\left(a_1, a_2, \cdots, a_k\right)}$. Here $r$ denotes the receiver of the asynchronous method call, $m$ is the name of the method called and $a_i$ stand for the arguments originally passed to the method.

\begin{sketch}
\begin{itemize}
\item why \lstinline!eval! expr was not implemented
\end{itemize}
\end{sketch}

\subsection{Predicates and Functions}\label{sct:pf}
\begin{sketch}
\begin{itemize}
	\item 1:1 correspondence between Chalice and SIL
	\item Explain global predicate- and function-Fractions
\end{itemize}
\end{sketch}

\subsection{Monitors with Deadlock Avoidance}\label{sct:mon}
\begin{sketch}
\begin{itemize}
\item Explain global monitor fraction
\item Too high-level for Mu: why establishing the correspondence between x.mu and waitlevel is difficult
\item Explain the solution: muMap, heldMap and the \$CurrentThread object.
\end{itemize}
\end{sketch}


% !TEX TS-program = xelatex+makeindex+bibtex
% !TEX encoding = UTF-8
% !TEX root = ../chalice2sil_report_klauserc.tex

\section{Evaluation}\label{sct:eval}

\subsection{SIL as a translation target/verification intermediate language}
One of the primary goals of writing Chalice2SIL was to gather experience working with SIL, both as a translation target and as a verification intermediate language. 

\subsubsection{Encoding of loops}
At the time we started this project, SIL did not have a dedicated \lstinline[language=Chalice]!while! loop node. 
Instead, programs would be encoded as a flat directed graph of basic blocks: a control-flow graph (CFG).
Loops were encoded as cycles with the ``backwards'' pointing edge explicitly marked (so that tools could traverse the CFG as an acyclic graph by ignoring those back edges).
Unfortunately, Silicon -- our verifier for SIL -- can currently only handle \lstinline[language=Chalice]!while! as they appear in Chalice. 
In those early days, Silicon would pattern match against the CFG to find \lstinline[language=Chalice]!while! loops and extract their components (condition, invariant, body), essentially lifting the program back up to the abstraction level of Chalice in terms of control flow.
If Chalice, which only supports \lstinline[language=Chalice]!while! loops, were the only source language that SIL ever had to support, this approach would have been fine.
But since the idea behind SIL was to eventually have multiple front ends, we decided to capture the fact that we currently can only verify \lstinline[language=Chalice]!while! loops in the language.
As a result a explicit loop node was added to the SIL control-flow graph.

\subsubsection{Syntactic distinction between assertions and program expressions}
Currently, SIL distinguishes between assertions (logic formulae and accessibility predicates) and program expressions on a syntactic level. 
Some language elements require assertions as operands (e.g., \lstinline[language=SIL]!exhale!) while others only accept program expressions (e.g., method arguments).
In the current implementation of the SIL abstract syntax tree (AST), there are two distinct and unrelated types: the type of assertions and the type of program expression.
Having the Scala compiler enforce that we never construct a SIL program where an accessibility predicate is used as a method argument is nice in theory, but proved to be more cumbersome than necessary in practice.

\begin{description}
\item[Only a partial solution] It is still possible to write translators that try to create illegal assertions and will fail due to runtime\footnote{In the context of $X$-to-SIL-translators, ``runtime'' refers to the execution of the translator.} checks built into the SIL AST.
While it would seem better to check as many properties statically as possible, only ending up with partial checks can result in a false sense of security. The SIL AST API is not exception-free and translators should be prepared deal with exceptions.
\item[Code duplication] Logical formulae and program expressions have a lot in common, e.g. logical operators or literal values. 
Distinguishing between a program-level \lstinline[language=Chalice]!true! and and an assertion-level \lstinline[language=Chalice]!true! has little benefit and at the same means that both producers and consumers of SIL need to have two pieces of code that handle boolean literals.
Since the Scala types of assertions and program expressions are unrelated, there is absolutely no opportunity for code reuse.
\item[Translation from Chalice] The Chalice compiler does not distinguish between assertions and program expressions on a syntactical level and instead enforces restrictions on where certain expressions can appear in semantic checks during type checking. Unfortunately, this makes syntax driven translation from Chalice to SIL highly ambiguous. The translator will come across many Chalice expressions where it is not a priori clear whether to translate them as SIL assertions or as SIL program expressions. For instance \ch{5 == 3} can be translated using the equality assertion \sil{\text{\lstinline[language=SIL]!5 == 3!}} or by first applying the integer equality domain function \lstinline!intEQ! and then lifting the resulting boolean program value up to assertion-level using the boolean domain predicate \lstinline!eval!: \sil{\text{\lstinline[language=SIL]!eval(intEQ(5,3))!}}.

Not all expressions can be translated either way and to find out which translation scheme is the correct one. 
This can mean that a translator needs to walk through Chalice expressions twice, either just trying both translation schemes in turn or first analysing the expression and then deciding on a translation scheme to use.

\item[Need to convert]
For Chalice2SIL it was sometimes necessary to convert between assertions and program expressions.
One example of this are expressions of the form \ch{\text{\lstinline[language=SIL]!old!}(e)} where $e$ translates to a SIL assertion. 
When a method with such an \lstinline[language=SIL]!old! expression in its precondition is forked, conceptually, the expression $e$ is evaluated and its ``value'' (\lstinline[language=Chalice]!true! or \lstinline[language=Chalice]!false!) stored in a field on the fork-token.
Because the right-hand side of an assignment needs to be a program expression in SIL, we first have to create a fresh boolean variable  and associate the truth value of that variable with the assertion:
\begin{lstlisting}[language=SIL]
var b : bool;
inhale eval(b) == e
\end{lstlisting}

This can always be done and thus making the translator (and later the verifier) jump through these hoops seems pointless.

\item[Naming]
In the actual implementation of the SIL AST, assertions are called \emph{expressions} and program expressions are called \emph{terms}. 
While the implementation uses this terminology very consistently, the terms fail to convey the key difference between assertions and program expressions, resulting in a lot of puzzled faces in conversations with people who are not intimately familiar with SIL's design.
\end{description}

\subsubsection{PTerms vs. DTerms vs. GTerms vs. Terms}
SIL makes another distinction on the syntactic level that we think is better handled as a semantic check.
To make sure that domain axioms don't contain AST nodes that are illegal in the context of a domain (such as references to heap locations), SIL has four types to represent program expressions.
\begin{description}
\item[DTerm] ``Domain'' terms represent the set of all program expressions that are legal in domain axiom specifications. References to quantifier variables are one example.
\item[PTerm] ``Program'' terms represent the set of all program expressions that are legal in actual program code (such as the right-hand side of assignment statements). Heap references are one example.
\item[GTerm] ``General' terms represent the set of all program expressions that are legal in \emph{all} contexts.  Integer literals are one example.
\item[Term] Represent the set of all program expressions. Examples include the full permission amount \lstinline[language=SIL]!write! or the permission mask lookup $\text{\lstinline[language=SIL]!perm!}\left(x.f\right)$.
\end{description}
Even though the SIL AST implementation uses Scala traits to capture the subset relationships between these sets, there is still an enormous amount of code duplication.
For instance, it is not enough to have one node type for domain function applications. 
Because you need to restrict the set from which the function application node draws its arguments, there is one domain function application node type for each of the four sets: \texttt{GDomainFunctionApplication}, \texttt{DDomainFunctionApplication}, \texttt{PDomainFunctionApplication} and \texttt{DomainFunctionApplication}.

Luckily, the subset types are all subtypes of \lstinline!DomainFunctionApplication! which allows for code reuse when \emph{consuming} (pattern matching) these data structures. 
When it comes to \emph{generating} theses node, however, we essentially had two options. 
One option was to duplicate a lot of code (e.g., have separate translation schemes for \lstinline!DDomainFunctionApplication! and \lstinline!PDomainFunctionApplication!).
We chose to translate to the most specific type whenever possible (e.g., if all arguments of a domain function application are \lstinline!DTerm!s themselves, create a \lstinline!DDomainFunctionApplication!) but return \lstinline!Term!s to accommodate for all Chalice program expressions.
In cases where \lstinline!PTerm!s are required instead of \lstinline!Term!s, we attempt to downcast to \lstinline!PTerm!.

The fact that even the SIL AST implementation itself employs this technique indicates that maybe in this case runtime checks (running during translation of a program to SIL) are better suited than trying to encode these constraints in the Scala type system.

The SIL AST implementation makes a similar distinction for assertions. 
There are \texttt{GExpression}s, \lstinline!DExpression!s, \lstinline!PExpression!s and \lstinline!Expression!s. 
We included a more complete grammar listing in appendix  \label{apdx:grammar}.

\subsubsection{Capturing state in SIL}
A pattern that often appears in the Boogie-based implementation of Chalice, is that one would make a copy of the heap and permission mask (both are ordinary variables from Boogie's perspective) then perform a series of operations and assertions on that copy (e.g., inhale, exhale).
Describing programs on a higher level of abstraction, SIL does not allow anything similar.
The \lstinline[language=Chalice]!perm! expression is about as close as we get to the permission mask.

Re-implementing the permission mask as we did to constrain the read fractions (section \ref{sct:meth}) seems incredibly wasteful since most tools that  work on SIL will have a concept of a permission map, maybe even specialised code to deal with them and all they see are manipulations of an abstract data structure (the map created by the Chalice2SIL translation), described by a couple of axioms.
So far this hasn't been a problem, though.

A similar issue is how we currently handle \lstinline[language=Chalice]!old! expression for fork/join.
What we would ideally like to do is to capture the program state at the point where a thread is forked off and then associate that state with the token.
Later when we \lstinline[language=Chalice]!join! on the token, we just evaluate the \lstinline[language=Chalice]!old! expression in the state associate with the token. 
We would not have to worry about the definedness of \lstinline[language=Chalice]!old! expressions at the point where we fork the token.

Chalice has other features where old state is being referenced. 
History constraints on monitor invariants and general \lstinline[language=Chalice]!eval! expressions are examples.

\subsection{Chalice2SIL+Silicon compared to Syxc}
\begin{sketch}
Compare Chalice2SIL+Silicon with Syxc (how complete is Chalice2SIL)
\end{sketch}




% !TEX TS-program = xelatex
% !TEX encoding = UTF-8
% !TEX root = ../chalice2sil_report_klauserc.tex
\clearpage
\section{Conclusion}\label{sct:conclusion}
As part of this project, we devised and implemented a translator from Chalice ASTs to SIL ASTs from scratch.
When the project started, not a single line of code existed for the SIL AST, Silicon or Chalice2SIL.
SIL itself was little more than a draft of its syntax on paper.
This turned out to be both a blessing and a curse.
On the one hand just about everyone we talked to had a slightly different idea of how a particular SIL construct was supposed to behave, at least initially.
On the other hand we had the opportunity to help shape SIL and the design of its AST.

While many parts of the translation from Chalice to SIL were comparatively straightforward due to the similarity between the two languages, some aspects of Chalice or its underlying permission model were surprisingly hard to implement within the constraints of SIL.
Overall, however, we think that the high-level design of SIL goes in the right direction.
Having  permission amounts as first class values and related constructs (\lstinline[language=SIL]!acc!, \lstinline[language=SIL]!exhale!, etc.) as built-in language constructs neatly decouples the representation of permissions in the verifier from the representation of the program to be verified.

With Chalice2SIL+Silicon we now have a first prototype of an automatic program verification tool-chain based on SIL, ready to be extended to include other tools that consume or transform SIL programs.


% !TEX TS-program = xelatex
% !TEX encoding = UTF-8
% !TEX root = ../chalice2sil_report_klauserc.tex

\appendix
\section{Full SIL Term and Expression Grammar}\label{apdx:grammar}
\begin{grammar}
<Expr> ::= 'acc' (<Location>, <Term>)
	\alt 'old' ( <Expr> )
	\alt 'unfolding' <Term>.<pred-id> 'by' <Term> 'in' <Expr>
	\alt <Term> == <Term>
	\alt <unary-op> <Expr>
	\alt <binary-op> <Expr>
	\alt <dom-pred-id>(\synrep{\synt{Term}}{,})
	\alt ∀ <logical-var-id> : <DataType> :: (<Expr>)
	\alt ∃ <logical-var-id> : <DataType> :: (<Expr>)
	\alt <GExpr>

<Location> ::= <Term>.<field-id>
	\alt <Term>.<pred-id>
\end{grammar}

\begin{grammar}
<PExpr> ::= 'acc' (<PLocation>, <PTerm>)
	\alt 'unfolding' <PTerm>.<pred-id> 'by' <PTerm> 'in' <PExpr>
	\alt <PTerm> == <PTerm>
	\alt <unary-op> <PExpr>
	\alt <binary-op> <PExpr>
	\alt <dom-pred-id>(\synrep{\synt{PTerm}}{,})
	\alt <GExpr>

<PLocation> ::= <PTerm>.<field-id>
	\alt <PTerm>.<pred-id>
\end{grammar}

\begin{grammar}
<DExpr> ::= <DTerm> == <DTerm>
	\alt <unary-op> <DExpr>
	\alt <binary-op> <DExpr>
	\alt <dom-pred-id>(\synrep{\synt{DTerm}}{,})
	\alt ∀ <logical-var-id> : <DataType> :: (<DExpr>)
	\alt ∃ <logical-var-id> : <DataType> :: (<DExpr>)
	\alt <GExpr>
\end{grammar}

\begin{grammar}
<GExpr> ::= <PExpr> == <PExpr>
	\alt <UnaryOp> <PExpr>
	\alt <BinaryOp> <PExpr>
	\alt <dom-pred-id>(\synrep{\synt{PExpr}}{,})
	\alt 'True'
	\alt 'False'
\end{grammar}

\begin{grammar}
<UnaryOp> ::= '¬'

<BinaryOp> ::= '∧' | '∨' | '$\equiv$' | '⇒'
\end{grammar}

\begin{grammar}
<Term> ::= 'if' <Term> 'then' <Term> 'else' <Term>
	\alt 'old'( <Term> )
	\alt <func-id>( \synrep{\synt{Term}}{,} )
	\alt <dom-func-id>( \synrep{\synt{Term}}{,} )
	\alt 'unfolding' <Term>.<pred-id> 'by' <Term> 'in' <Term>
	\alt (<Term>) : <DataType>
	\alt <Term>.<field-id>
	\alt 'perm'( <Location> )
	\alt 'write'
	\alt '0'
	\alt  'E'
	\alt <GTerm>
\end{grammar}

\begin{grammar}
<PTerm> ::= 'if' <PTerm> 'then' <PTerm> 'else' <PTerm>
	\alt <var-id>
	\alt <func-id>( \synrep{\synt{PTerm}}{,} 
	\alt <dom-func-id>( \synrep{\synt{PTerm}}{,} )
	\alt 'unfolding' <PTerm>.<pred-id> 'by' <PTerm> 'in' <PTerm>
	\alt (<PTerm>) : <DataType>
	\alt <PTerm>.<field-id>
	\alt <GTerm>
\end{grammar}

\begin{grammar}
<DTerm> ::= 'if' <DTerm> 'then' <DTerm> 'else' <DTerm>
	\alt <logical-var-id>
	\alt <dom-func-id>( \synrep{\synt{DTerm}}{,} )
	\alt <GTerm>
\end{grammar}

\begin{grammar}
<GTerm> ::= 'if' <GTerm> 'then' <GTerm> 'else' <GTerm>
	\alt <integer-literal>
	\alt 	\alt <dom-func-id>( \synrep{\synt{GTerm}}{,} )
\end{grammar}

\bibliographystyle{alpha}
\bibliography{lib/chalice-related}

\end{document}
