% !TEX TS-program = xelatex
% !TEX encoding = UTF-8

\documentclass[11pt]{article} % use larger type; default would be 10pt

\usepackage[fleqn]{amsmath}
\usepackage{amsthm,amsfonts}
\usepackage{amssymb} % Must be included BEFORE unicode!!!
\usepackage{fontspec} % Font selection for XeLaTeX; see fontspec.pdf for documentation
\defaultfontfeatures{Mapping=tex-text} % to support TeX conventions like ``---''
\usepackage{xunicode} % Unicode support for LaTeX character names (accents, European chars, etc)
\usepackage{xltxtra} % Extra customizations for XeLaTeX
%\usepackage[math-style=ISO]{unicode-math}
%\usepackage{multicol}


\setmainfont{Cambria} % set the main body font (\textrm), assumes Charis SIL is installed
\setsansfont{Calibri}
\setmonofont{Consolas}
%\setmathfont{Cambria Math}

% other LaTeX packages.....
%\usepackage{savetrees}
\usepackage{fullpage}
%\usepackage{bussproofs}
\usepackage{fancyhdr}
\usepackage{listings}
\usepackage{color}

% =====================================================
% ===== META INFORMATION ==============================
% =====================================================

\newcommand{\metaSemester}{HS11}
\newcommand{\metaCourseFull}{Translating Chalice into SIL}
\newcommand{\metaCourseShort}{SIL}
\newcommand{\metaCourse}{\metaCourseFull (\metaCourseShort)}
\newcommand{\metaSectionPrefix}{}
\newcommand{\metaAuthor}{Christian Klauser}
\newcommand{\metaAuthorNethz}{klauserc}
\newcommand{\metaAuthorLegi}{08-919-490}

\newcommand{\metaTitle}{Problem Description}

% =====================================================
% =====================================================

\pagestyle{fancy}%
\renewcommand{\headrulewidth}{0pt}%
\renewcommand{\footrulewidth}{0pt}%
%\setlength{\headheight}{14pt}%
%\setlength{\footskip}{10pt}%
\fancyhead[L]{}%
\fancyhead[R]{}%
\fancyfoot[C]{\thepage}%

\usepackage{graphicx} % support the \includegraphics command and options
\usepackage{pdflscape}

%=========================================

\newcommand{\abs}[1]{{\left\lvert#1\right\rvert}}
\newcommand{\norm}[1]{{\left\lVert#1\right\rVert}}
\newcommand{\setw}[2]{\ensuremath{\left\{#1\:\middle|\:#2\right\}}}
\newcommand{\albool}{\ensuremath{\{0,1\}}}
\newcommand{\alab}{\ensuremath{\{a,b\}}}
\newcommand{\NN}{\ensuremath{\mathbb{N}}}
\newcommand{\RR}{\ensuremath{\mathbb{R}}}
\newcommand{\ZZ}{\ensuremath{\mathbb{Z}}}
\newcommand{\QQ}{\ensuremath{\mathbb{Q}}}
\newcommand{\CC}{\ensuremath{\mathbb{C}}}
\newcommand{\KK}{\ensuremath{\mathbb{K}}}
\newcommand{\MM}{\ensuremath{\mathbb{M}}}
\newcommand{\E}[1]{\ensuremath{\mathbf{E}\left[#1\right]}}
\newcommand{\ECond}[2]{\ensuremath{\mathbf{E}\left[#1\:\middle|\:#2\right]}}
\renewcommand{\Pr}[1]{\ensuremath{\mathbf{Pr}\left[#1\right]}}
\newcommand{\Prs}[2]{\ensuremath{\mathbf{Pr}_{#2}\left[#1\right]}}
\newcommand{\PrCond}[2]{\ensuremath{\mathbf{Pr}\left[#1\:\middle|\:#2\right]}}

\newcommand{\entails}{\ensuremath{\vdash\ }}
\renewcommand{\implies}{\ensuremath{\ \rightarrow\ }}

%=========================================

\usepackage[parfill]{parskip} 
\usepackage{titlesec}
%\titleformat{\section}[display]{\normalfont\Large\bfseries}{Problem \thesection:}{1em}{}
\titleformat{\section}{\normalfont\Large\bfseries\sffamily}{\metaSectionPrefix \thesection\ }{0.5em}{}
\titleformat{\subsection}{\normalfont\large\bfseries\sffamily}{\thesubsection}{0.5em}{}
\titleformat{\subsubsection}{\normalfont\bfseries\sffamily}{\thesubsubsection}{0.4em}{}
\renewcommand{\thesection}{\arabic{section}}
\renewcommand{\thesubsection}{(\alph{subsection})}
\renewcommand{\thesubsubsection}{(\roman{subsubsection})}

%=========================================

\title{{\Huge Translating Chalice into SIL}\\{\small\ }\\{\huge Problem Description}}
\author{Christian Klauser\\ \texttt{klauserc@student.ethz.ch}}
%\date{} % Activate to display a given date or no date (if empty),
         % otherwise the current date is printed 

\begin{document}
\maketitle

\section{Background}
Chalice is an experimental programming language designed for specifying and verifying concurrent programs. It is centred on the idea of passing permissions back and forth between methods, threads, monitors and channels. Currently, the Chalice verifier works by encoding its input program and Chalice's semantics into a Boogie program and having that program verified by the Boogie verifier. 
SIL is the intermediate language used in viper, a long term project aiming to create an extensible, symbolic-execution-based verifier for Scala.

\section{Main task}
The goal of this Bachelor's thesis is to design and implement a new back end for the Chalice compiler, targeting SIL instead of Boogie. In addition to interfacing Chalice with a new verifier backend, the project helps flesh out SIL and test the tool chain of the viper project.

The resulting program should
\begin{itemize}
\item use the existing Chalice front end with as few changes as possible. This ensures that the language does not get fragmented.
\item be able to translate basic language constructs like assignments, conditions and loops so that they can be verified.
\item be able to translate method calls, thread forks and joins, making sure that permissions are passed around correctly.
\item be able to translate monitor-based locks, including transfer of permissions into and out of the lock. Chalice's deadlock-avoidance mechanism can be considered as an extension.
\item track locations in the original source code in order to generate meaningful error messages in case an input program fails verification. The use of an intermediate representation should be as transparent as possible.
\end{itemize}

\section{Extensions}
Depending on the progress of the main task, several extensions to the project are possible. 
\begin{itemize}
\item Chalice comes with support for \emph{predicates} and \emph{functions}, features that allow for information hiding and code reuse in invariants, pre- and postconditions. \emph{Functions} in particular offset Chalice's lack of \emph{pure} methods. \emph{Predicates} and \emph{functions} need to be translated into the target language in order to verify Chalice programs using these features.
\item Deadlock-avoidance is a very important feature for a language that is designed to enable the verification of concurrent programs. Chalice already provides a mechanism for preventing deadlocks on \emph{monitors}; it would just have to be translated into the target language.
\item Many of the more interesting Chalice programs make use of \emph{channels} to simplify communication between threads (Actor model). Like with \emph{monitors}, there is a mechanism for avoiding deadlocks on \texttt{receive} operations. Again, these would have to be translated into the target language.
\end{itemize}

\end{document}
